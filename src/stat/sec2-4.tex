\section{Функция правдоподобия. Достаточные статистики, полные статистики. Теорема факторизации}

В зависимости от типа семейства распределения $\Dist_\theta$ обозначим через $f_{\theta}(y)$ одну из следующих функций:
\begin{equation*}
    f_{\theta}(y) =
    \left\{\begin{array}{ll}
    \text { плотность } f_{\theta}(y), & \text { если } \Dist_{\theta} \text { абсолютно непрерывно, } \\
    P_{\theta}\left(X_{1}=y\right), & \text { если } \Dist_{\theta} \text { дискретно. }
    \end{array}\right.
\end{equation*}

\begin{defn}
    \textit{Функция правдоподобия} выборки $\SampleX$:
    \begin{equation*}
        L(\SampleX, \theta)=f_{\theta}\left(X_{1}\right) \cdot f_{\theta}\left(X_{2}\right) \cdot \ldots \cdot f_{\theta}\left(X_{n}\right)=\prod_{i=1}^{n} f_{\theta}\left(X_{i}\right).
    \end{equation*}
\end{defn}

В дискретном случае функция правдоподобия принимает вид:
\begin{equation*}
    \begin{aligned}
        L(\SampleX, \theta)=\prod_{i=1}^{n} f_{\theta}(x_{i}) 
        = \MyPrTh (X_{1}=x_{1}) \cdot \ldots \cdot \MyPrTh (X_{n}=x_{n}) = \\
        = \MyPrTh (X_{1}=x_{1}, \ldots, X_{n}=x_{n}).
    \end{aligned}
\end{equation*}

\begin{rmrk}
    Функция правдоподобия зависит от выборки~$\SampleX$ и от параметра~$\theta$.
    Если мы зафиксируем некоторое~$\theta_0$, то получим вероятностную меру над выборочным пространством~$\Real^n$.
    Тогда значение функции правдоподобия~$L(\mathbf{x}, \theta_0)$ на некоторой реализации выборки~$x$~--- это вероятность этой реализации в предположении, что параметр~$\theta_0$~--- истинное значение. \\
    Если же мы напротив, зафиксируем некоторую реализацию выборки~${\mathbf{x}^* = (x_1^*, \ldots, x_n^*)}$ (что соответствует реальным условиям~--- как правило, у нас есть некоторые наблюдения, а параметр неизвестен), то значение функции правдоподобия на некотором параметре~$\theta$~--- это "<правдоподобие">, "<вероятность"> этого параметра.
    Но, строго говоря, при фиксированной реализации выборки~$x^*$ функция правдоподобия~$L(\mathbf{x}^*, \theta)$ не является вероятностной мерой над~$\Theta$.
    Рассмотрим пример.
\end{rmrk}

\begin{exmp}
    При двух последовательных независимых подбрасываниях монетки выпало два "<орла">.
    Запишем функцию правдоподобия выборки $\SampleX= \bigl(X_1, X_2\bigr), \; X_i \in \{0, 1\}$.
    Обозначим вероятность выпадения "<орла"> ${\bigl(\MyPrTh(X_i = 1)\bigr)}$ за $\theta, \; \theta \in [0, 1]$.
    Тогда
    \begin{equation*}
        L(\SampleX, \theta) = \MyPrTh (X_1 = x_1) \cdot \MyPrTh (X_2 = x_2).
    \end{equation*}
    Для любого фиксированнного значения параметра $\theta_0$ функция правдоподобия $L(\SampleX, \theta_0)$ является функцией совместного распределения $(X_1, X_2)$:
    \begin{center}
        \begin{tabular}{|c|c|c|}
            \hline
            & $X_1 = 1$ & $X_1 = 0$ \\
            \hline
            $X_2 = 1$ & $\theta^2_0$ & $(1-\theta_0)\theta_0$ \\
            \hline
            $X_2 = 0$ & $\theta_0(1-\theta_0)$ & $(1-\theta_0)^2$ \\
            \hline
        \end{tabular}
    \end{center}
    \begin{gather*}
        \sum\limits_{x_1 = 0}^{1} \sum\limits_{x_2 = 0}^{1} L(\mathbf{x}, \theta_0) = 
        \sum\limits_{x_1 = 0}^{1} \sum\limits_{x_2 = 0}^{1} \MyPr{\theta_0} (X_1 = x_1) \cdot \MyPr{\theta_0} (X_2 = x_2) = \\
        = \theta_0^2 + (1 - \theta_0)\theta_0 + \theta_0(1-\theta_0) + (1 - \theta_0)^2 
        = \theta_0^2 + 2\theta_0 - 2\theta_0^2 + (1 - 2\theta_0 + \theta_0^2)
        = 1.
    \end{gather*}
    Но если мы зафиксируем реализацию выборки $x^* = (x_{1}^{*} = 1, x_{2}^{*} = 1)$ и проинтегрируем по $\Theta = [0, 1]$, то получим
    \begin{equation*}
        \int\limits_0^1 L(\mathbf{x}^*, \theta) \, d\theta = 
        \int\limits_0^1 \MyPrTh(X_1 = 1) \cdot \MyPrTh(X_2 = 1) \, d\theta = 
        \int\limits_0^1 \theta^2 \, d\theta = 
        \left. \frac{\theta^3}{3} \right|_0^1 = 
        \frac{1}{3}.
    \end{equation*}

    Выходит, что $L(\mathbf{x}^*, \theta)$ не может быть вероятностной мерой, так как интеграл по $\Theta$ не равен 1.
    %А $L(\SampleX, \theta_0)$ определяет совместное распределение случайных величин $(X_1, X_2)$: если они абсолютно непрерывны~--- то это совместная плотность, а если дискретны~--- то вероятность принятия соответствующих значений.
\end{exmp}

%Таким образом, смысл функции правдоподобия~--- <<вероятность>>\footnote{Строго говоря, функция правдоподобия не является вероятностной мерой над $\theta$, хотя бы потому, что $\int\limits_{\Theta} L(\mathbf{x}, \theta) d\theta \neq 1$. 
%Термин <<вероятность>> применён здесь в переносном смысле.} попасть в заданную точку при соответствующем параметре $\theta$ в дискретном случае; для абсолютно непрерывного аналогично~--- вероятность попасть в куб с центром в $x_1, \ldots, x_n$ и сторонами $dx_1, \ldots, dx_n$.

\vspace{5mm}
\begin{defn}
    \textit{Достаточная статистика}~--- статистика $T(\SampleX)$ такая, что $\forall t \in \Real^m,$~\footnote{Напомним, что вообще говоря, статистика~--- 
    это отображение $T(X)\colon \Real^n \mapsto \Real^m$ } $ \forall B \in \Borel^n$ условное распределение $\MyPrTh \Bigl(\Sample \in B | \, T=t\Bigr)$ не зависит от параметра $\theta$. Иными словами, статистика $T(\SampleX)$ называется достаточной, если $\ExpTh \bigl( \SampleX\, | \, T(\SampleX) \bigr)$ не зависит от $\theta$.
\end{defn}

Иными словами, достаточная статистика содержит в себе всю информацию о параметре, которую можно извлечь из выборки.

Достаточная статистика всегда существует~--- например, сама выборка $\SampleX$ является достаточной статистикой.
Однако этот тривиальный пример не очень полезен.
Как правило, нас интересуют достаточные статистики малой размерности.
Для того, чтобы проверить, является ли статистика достаточной, можно использовать критерий факторизации.
\begin{namedthm}[Критерий факторизации]
    $T(\SampleX)$~--- достаточная статистика $\iff$ её функция правдоподобия представима в виде 
    \begin{equation*}
        L(X_{1}, \ldots, X_{n} , \theta) \stackrel{\text{п.н.}}{=} g\bigl(T(\SampleX), \theta\bigr) \cdot h(\SampleX)
    \end{equation*}
\end{namedthm}

\begin{proof}
    Рассмотрим только дискретный случай. 
    \begin{enumerate}
        \item[$\implies$] Пусть $T(\SampleX)$~--- достаточная статистика. 
            Возьмём произвольную реализацию выборки $\boldsymbol{x}$ и обозначим $t = T(\boldsymbol{x})$. Тогда
            \begin{multline*}
                L(\boldsymbol{x}, \theta) = \MyPrTh (\SampleX=\boldsymbol{x})=\MyPrTh (\SampleX=\boldsymbol{x}, T(\SampleX)=t) =\\
                = \underbrace{\MyPrTh (T(\SampleX)=t)}_{g(T(\SampleX), \theta)} \underbrace{\MyPrTh (\SampleX=\boldsymbol{x} | T(\SampleX)=t)}_{h(\SampleX)}
            \end{multline*}
        \item[$\impliedby$] Пусть теперь функция правдоподобия имеет вид $L(\boldsymbol{x}, \theta)=g\bigl(T(\boldsymbol{x}), \theta\bigr) h(\boldsymbol{x})$. 
            Тогда, если $x$ таково, что $T(\boldsymbol{x})=t$, то:
            \begin{multline*}
                \MyPrTh (\SampleX=\boldsymbol{x} | \, T(\SampleX)=t) =\frac{\MyPrTh (\SampleX=\boldsymbol{x}, T(\SampleX)=t)}{\MyPrTh (T(\SampleX)=t)}
                =\frac{\MyPrTh (\SampleX=\boldsymbol{x})}{\sum\limits_{\boldsymbol{x}^{\prime}: T(\boldsymbol{x}^{\prime})=t} \MyPrTh(\SampleX=\boldsymbol{x}^{\prime})} = \\
                = \frac{g(t, \theta) h(\boldsymbol{x})}{\sum\limits_{\boldsymbol{x}^{\prime}: T(\boldsymbol{x}^{\prime})=t} g(t, \theta) h(\boldsymbol{x}^{\prime})}
                = \frac{h(\boldsymbol{x})}{\sum\limits_{\boldsymbol{x}^{\prime}: T(\boldsymbol{x}^{\prime})=t} h(\boldsymbol{x}^{\prime})}
            \end{multline*}
            Т.е. вероятность не зависит от $\theta$, а значит, $T(\SampleX)$ достаточная.
    \end{enumerate}
\end{proof}

\begin{defn}
    Статистика $T(\SampleX)$ называется \textit{полной}, если для любой борелевской функции $\varphi(T)$, для которой $\ExpTh \varphi(T)=0~\AllTh$, справедливо: $\varphi(T) \stackrel{\text{п.н.}}{=}0$.
\end{defn}

\begin{exmp}
    Рассмотрим равномерное распределение $\Uniform_{[0,\theta]}$ и докажем, что статистика $T(\SampleX) = X_{(n)}$ является полной. 
    Напоминаем, что плотность распредения $X_{(n)}$ для $U[0, \theta]$ - это $f(t) = \cfrac{n t^{n-1}}{\theta^n} \, \Ind(0 \leqslant t \leqslant \theta)$.
    \begin{gather*}
        \ExpTh \varphi(X_{(n)})= \int\limits_{0}^{\theta} \varphi(t) \cfrac{nt^{n-1}}{\theta^n} dt = 0 \implies \int\limits_{0}^{\theta} \varphi(t) t^{n-1} dt = 0 \implies \\
        \{ \text{продифференцируем по}~ \theta\} \implies 
        \varphi(\theta) \theta^{n-1} = 0  \implies \varphi(T) \equiv 0 ~~ \forall T > 0
    \end{gather*}
    Таким образом, $\varphi(T) \stackrel{\text{п.н.}}{=} 0$ при $T \geqslant 0$, и указанная статистика является полной.
\end{exmp}
