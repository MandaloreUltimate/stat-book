\chapter{Математическая статистика}

\section{Статистическая структура. Выборка. Статистика. Порядковые статистики. Вариационный ряд. Эмпирическая функция распределения}

\begin{defn}
    \textit{Статистическая структура}~--- совокупность $(\Real^n, \Borel^n, \Dist_\theta)$, 
    где $\Real^n$~--- выборочное пространство, 
    $\Borel^{n}$~--- борелевская $\sigma$-алгебра на $\Real^n$, $\Dist_\theta$~--- семейство распределений, 
    определённых на $\Borel^n$, параметризованное одно- или многомерным числовым параметром: 
    $\Dist_\theta = (\MyPr{\theta} \colon \theta \in \Theta \subseteq R^{m})$.
\end{defn}

\begin{defn}
    \textit{Выборка} $\SampleX= \Sample$ объёма $n$~--- набор из $n$ независимых и одинаково распределённых случайных величин\footnote{Вообще говоря, в приложениях возникают также выборки, состоящие из зависимых или разнораспределённых элементов, но изучение их свойств не входит в этот курс.}, имеющих такое же распределение, как и наблюдаемая случайная величина $\xi$.
\end{defn}

До того, как эксперимент проведён, выборка~--- набор случайных величин, после~--- набор чисел из множества возможных значений случайной величины. 
Числовой набор $\boldsymbol{x} = (x_1, \ldots, x_n)$ будем называть \textit{реализацией выборки}.

\begin{rmrk}
    Статистическая структура очень похожа на многомерное \hyperlink{induced_prob_space}{\textit{индуцированное вероятностное пространство}}.
    Отличие заключается в том, что если ранее мы использовали распределение лишь одной случайной величины~$\xi$, то здесь используется целое семейство.
    Почему?
    Дело в том, что мы ещё не знаем, каковы параметры исследуемого распределения.
    Задача математической статистики как раз и заключается в том, чтобы их найти, основываясь на наблюдениях~--- реализациях выборок.

    Отметим также то, что в различных источниках $\Dist_\theta$ может называться семейством \textit{вероятностных мер}.
    Это тоже верно, но может внести путаницу~--- распределение случайной величины действительно является мерой, определённой на борелевской $\sigma$-алгебре.
    Однако вероятностная мера, вообще говоря, не связана ни с какой случайной величиной~--- это просто отображение из некоторой $\sigma$-алгебры событий в вещественные числа, удовлетворяющее аксиомам неотрицательности, ограниченности и счётной аддитивности.
    А распределение случайной величины~--- это композиция прообраза и вероятностной меры: $P_{\xi} = \MyPr \circ \xi^{-1}$.
    Т.е. распределение содержит информацию о конкретной случайной величине, с некоторыми фиксированными параметрами.
    Именно поэтому мы используем семейство распределений.
    Нам не нужно много вероятностных мер~--- нам нужно много распределений с разными наборами параметров.

    Следует обратить внимание на то, что поскольку распределение зависит от параметра $\theta$, то от него будут зависеть и математическое ожидание, и дисперсия, и прочие числовые характеристики налюдаемой случайной величины.
    Например, для "<честной"> монетки математическое ожидание при 10 подбрасываниях должно быть равно 5, а дисперсия~--- $npq = 10 \cdot \frac{1}{2} \cdot \frac{1}{2} = 2.5$. 
    Но в случае асимметричной монетки с вероятностью выпадения "<решки">, равной $0.7$, математическое ожидание будет равно 7, а дисперсия~--- $2.1$.
    Поэтому лучше явно указывать зависимость от параметра и писать $\MyPrTh\,, \ExpTh, \VarTh$.
\end{rmrk}

\begin{defn}
    \textit{Статистика} или \textit{оценка}~--- измеримая функция от выборки, не зависящая от любых других параметров. Чаще всего статистики используются для поиска неизвестного параметра распределения $\theta$ и имеют вид $T\colon \Real^n \mapsto \Theta$.
\end{defn}

\begin{defn}
    \textit{Вариационный ряд}~--- набор случайных величин $X_{(1)}, \ldots, X_{(n)}$, который получается при упорядочении выборки $\SampleX= \Sample$ по возрастанию на каждом элементарном исходе. 

    $X_{(1)}(\omega)=\min \bigl(X_{1}(\omega), \ldots, X_{n}(\omega)\bigr)$~--- \textit{минимальная порядковая статистика}, 
    $X_{(n)}(\omega)=\max \bigl(X_{1}(\omega), \ldots, X_{n}(\omega)\bigr)$~--- соответственно, \textit{максимальная}.
    Элемент $X_{(k)}$~--- \textit{$k$-я порядковая статистика}. 
\end{defn}

\begin{rmrk}
    Согласно нашему определению, вариационный ряд \textit{не является выборкой}, хотя бы потому, что разные порядковые статистики, как мы вскоре убедимся, имеют разное распределение.
    Однако он является статистикой $T\colon \Real^n \mapsto \Real^n$, так как зависит только от выборки\footnote{Конечно, формально ещё необходимо проверить на измеримость функцию, осуществляющую переупорядочение}.
    Порядковые статистики тоже удовлетворяют определению статистики, но являются уже отображениями из $\Real^n$ в $\Real$.
\end{rmrk}

\begin{defn}
    \textit{Эмпирическая функция распределения}, построенная по выборке $X_{1}, \ldots, X_{n}$ объёма $n$,~--- случайная функция $F_{n}^{*}$:
    \begin{equation*}
        F_{n}^{*}(y) =\frac{1}{n} \sum\limits_{i=1}^{n} \Ind(X_{i}<y) \quad \forall y \in \Real
    \end{equation*}
\end{defn}

Эмпирическая функция распределения строится по вариационному ряду следующим образом:

\begin{equation*}
    F_{n}^{*}(y)=\left\{\begin{array}{ll}
    0,   & y \leqslant X_{(1)} \\
    k/n, & X_{(k)} < y \leqslant X_{(k+1)} \\
    1,   & y > X_{(n)}
    \end{array}\right.
\end{equation*}

\begin{exmp}
    Найдём эмпирические функции распределения для крайних порядковых статистик.
    \begin{gather*}
        \begin{aligned}
            F_{(1)}(x)=\MyPrTh \bigl(X_{(1)} < x \bigr) 
        = 1 - \MyPrTh \bigl(X_{(1)} \geqslant x \bigr) 
        = 1 - \MyPrTh \bigl(x_{1} \geqslant x, \ldots, x_{n} \geqslant x\bigr) = \\
        = 1 - \prod_{i=1}^{n} \MyPrTh (x_{i} \geqslant x) 
        = 1 - \bigl(\MyPrTh ({x}_{1} \geqslant x)\bigr)^{n} 
        = 1 - \bigl(1 - F_{\theta}(x)\bigr)^{n}\!.
        \end{aligned} \\
        \begin{aligned}
            F_{(n)}(x) 
            = \MyPrTh \bigl(X_{(n)} < x\bigr) 
            = \MyPrTh \bigl(x_{1} < x, \ldots, x_{n} < x\bigr) = \\
            = \prod_{i=1}^{n} \MyPrTh (x_{i} < x) 
            = \bigl(\MyPrTh ({x}_{1} < x)\bigr)^{n} 
            = \bigl( F_{\theta}(x) \bigr)^n \!.
        \end{aligned}
    \end{gather*}
\end{exmp}

\begin{namedthm}[Свойства эмпирической функции распределения]\leavevmode
    \begin{enumerate}
        \item Пусть $\Sample$~--- выборка из семейства распределений $\Dist_{\theta}$ с функцией распределения $F_{\theta}$ и пусть $F_{n}^{*}$ — эмпирическая функция распределения, построенная по этой выборке. 
        Тогда $F_{n}^{*}(y) \xrightarrow[n \to \infty]{\text{p}} F_{\theta}(y)$ для любого $y \in \Real$ и $\AllTh.$
        \item Для любого y $\in \Real$ и любого $\theta \in \Theta$:
        \begin{enumerate}[label={\arabic*)}]
            \item $\ExpTh F_{n}^{*}(y) = F_{\theta}(y)$, т.е. $F_{n}^{*}(y)$~--- несмещённая оценка для $F_{\theta}(y)$.
            \item $\VarTh F_{n}^{*}(y)=\cfrac{F_{\theta}(y)\bigl(1-F_{\theta}(y)\bigr)}{n} \: \leqslant \: \cfrac{1}{4n}$\, .
            \item Пусть $\sigma^2(y) = \bigl(1 - F_{\theta}(y)\bigr)F_{\theta}(y)$. 
            Тогда
            \begin{equation*}
                \sqrt{n}\bigl( F_{n}^{*}(y)-F_{\theta}(y) \bigr) \; \xrightarrow[n \to \infty]{\text{d}} \; \Normal_{0, \sigma^2(y)},
            \end{equation*}
            т.е. $F_{n}^{*}(y)$~--- асимптотически нормальная оценка для $F_{\theta}(y)$.
            \item $n F_{n}^{*}(y) \sim \Binom_{n, F_{\theta}(y)}.$
            \item $F_{n}^{*}(y) \xrightarrow[n \to \infty]{\text{п.н.}} F_{\theta}(y).$
        \end{enumerate}
    \end{enumerate}
\end{namedthm}

\begin{proof}\leavevmode
    \begin{enumerate}
        \item 
            $F_{n}^{*}(y)=\frac{1}{n} \sum\limits_{i=1}^{n} \Ind(X_{i}<y)$, при этом случайные величины $\Ind(X_{1}<y)$, $\Ind(X_{2}<y), \ldots$ независимы и одинаково распределены, их математическое ожидание конечно:
            \begin{equation*}
                \ExpTh \Ind(X_{1}<y)=1 \cdot \MyPrTh (X_{1}<y)+0 \cdot \MyPrTh (X_{1} \geqslant y) = \MyPrTh (X_{1}<y)=F_{\theta}(y) < \infty
            \end{equation*}
            Следовательно, можно применить ЗБЧ в форме Хинчина:
            \begin{equation*}
                F_{n}^{*}(y)=\cfrac{\sum\limits_{i=1}^{n} \Ind(X_{i}<y)}{n} \xrightarrow[n \to \infty]{\text{p}} \ExpTh \Ind(X_{1}<y)=F_{\theta}(y) \quad \AllTh.
            \end{equation*}
        \item 
            Заметим:
            \begin{gather*}
                \Ind(X_{1}<y) \sim  \Binom_{1, F_{\theta}(y)} \implies \ExpTh \Ind(X_{1}<y) = F_{\theta}(y), \\
                \VarTh \Ind(X_{1}<y) = F_{\theta}(y)(1-F_{\theta}(y)) \quad \AllTh.
            \end{gather*}
            \begin{enumerate}[label={\arabic*)}]
                \item Случайные величины $\Ind(X_{i}<y)$ одинаково распределены, поэтому:
                \begin{equation*}
                    \ExpTh F_{n}^{*}(y) = \ExpTh \, \cfrac{\sum\limits_{i=1}^{n} \Ind(X_{i}<y)}{n} =\cfrac{\sum\limits_{i=1}^{n} \ExpTh \Ind(X_{i}<y)}{n}=\cfrac{n \ExpTh \Ind(X_{1}<y)}{n}=F_{\theta}(y)  
                \end{equation*}
                
                \item Случайные величины $\Ind(X_{i}<y)$ независимы и одинаково распределены, поэтому:
                \begin{multline*}
                    \VarTh F_{n}^{*}(y)
                    = \VarTh \, \cfrac{\sum\limits_{i=1}^{n} \Ind(X_{i}<y)}{n}
                    = \cfrac{\sum\limits_{i=1}^{n} \VarTh \Ind(X_{i}<y)}{n^{2}}
                    = \\
                    = \cfrac{n\VarTh \Ind(X_{1}<y)}{n^{2}}
                    = \cfrac{F_{\theta}(y)\bigl(1-F_{\theta}(y)\bigr)}{n}
                \end{multline*}
                Значения $F_{\theta}(y)$ принадлежат отрезку $[0, 1]$, а значит, произведение $F_{\theta}(y)\bigl(1 - F_{\theta}(y)\bigr) \leqslant \cfrac{1}{2}\cdot\left(1 - \cfrac{1}{2}\right) = \cfrac{1}{4}~$ (нетрудно убедиться, что 1/2~--- точка максимума). 
                А значит, $\VarTh F_{n}^{*} \leqslant \cfrac{1}{4n}\,$. 
            \end{enumerate}
        \item 
            Применим ЦПТ:
            \begin{multline*}
                \sqrt{n}\bigl( F_{n}^{*}(y)-F_{\theta}(y) \bigr)
                = \sqrt{n}\left(\cfrac{\sum \Ind(X_{i}<y)}{n}-F_{\theta}(y)\right) 
                = \\
                = \cfrac{\sum\limits_{i=1}^{n} \Ind(X_{i}<y)-n F_{\theta}(y)}{\sqrt{n}} 
                = \cfrac{\sum\limits_{i=1}^{n} \Ind(X_{i}<y)-n \ExpTh\Ind(X_{1}<y)}{\sqrt{n}} 
                \xrightarrow[n \to \infty]{\text{d}} \\
                \xrightarrow[n \to \infty]{\text{d}} \Normal_{0, \VarTh\Ind(X_{1}<y)}
                = \Normal_{0, (1-F_{\theta}(y))F_{\theta}(y)}.
            \end{multline*}
        \item 
            Следует из устойчивости по суммированию биномиального распределения. 
            Поскольку $\Ind\left(X_{i}<y\right)$ независимы и имеют биномиальное распределение $\Binom_{1, F_{\theta}(y})$, то их сумма
            \begin{equation*}
                n F_{n}^{*}(y)=\Ind\left(X_{1}<y\right)+\ldots+\Ind\left(X_{n}<y\right)
            \end{equation*}
            имеет биномиальное распределение $\Binom_{n, F_{\theta}(y})$.
            
        \item 
            Выберем произвольный $y \in \Real$. 
            $\xi_i = \Ind(X_i < y)$ независимы, одинаково распределены и $\exists \ExpTh \xi_i = F_{\theta}(y)$.
            Тогда можно применить \hyperlink{SLLN}{усиленный закон больших чисел в форме Колмогорова}: ${\MyPrTh\left(\lim\limits_{n \to \infty} \frac{1}{n} \sum\limits_{i = 1}^{n}\xi_i = F_{\theta}(y)\right) = 1} \; \AllTh$, что является непосредственным определением сходимости эмпирической функции распределения почти наверное к теоретической.
    \end{enumerate}  
\end{proof}
  % Статистическая структура. Выборка. Статистика. Порядковые статистики. Вариационный ряд. Эмпирическая функция распределения
\section{Точечная оценка. Несмещённость, состоятельность, оптимальность. Теорема о единственности оптимальной оценки}
\begin{defn}
    \textit{Статистика} или \textit{оценка} $T(\SampleX)$~--- измеримая функция от выборки.
\end{defn}

\begin{defn}
    \textit{Несмещённая оценка} параметра $\theta$~--- статистика $T(\SampleX)$, т.ч. $\AllTh\colon \ExpTh T(X) = \theta$.
\end{defn}

Обозначим $T_n(\SampleX) = T(\SampleX)$, чтобы подчеркнуть зависимость от объёма выборки.
\begin{defn}
    \textit{Асимптотически несмещённая оценка} параметра $\theta$~--- статистика $T_n(\SampleX)$, т.ч. $\AllTh \colon \; \ExpTh T_n(\SampleX) \xrightarrow[n \to \infty]{} \theta$.
\end{defn}

\begin{defn}
    \textit{Состоятельная оценка} параметра $\theta$~--- статистика $T_n(\SampleX)$, т.ч. $\AllTh \colon \; T_n(\SampleX) \xrightarrow[n \to \infty]{p} \theta$.
\end{defn}

Оценки также могут вводиться и для функций $\tau(\theta)$ параметра $\theta$, для них все вышеуказанные определения вводятся аналогично.

Несмещённость означает отсутствие ошибки «в среднем», т. е. при систематическом использовании данной оценки. 
Несмещённость является желательным, но не обязательным свойством оценок. 
Достаточно, чтобы смещение оценки (разница между её средним значением и истинным параметром) уменьшалось с ростом объёма выборки. 
Поэтому асимптотическая несмещённость является весьма желательным свойством оценок. 

Свойство состоятельности означает, что последовательность оценок приближается к неизвестному параметру при увеличении количества наблюдений. 
Вспомним определение сходимости по вероятности: ${\MyPrTh \Bigl(\bigl| T_n(\SampleX) - \theta \bigr| > \varepsilon \Bigr) \xrightarrow[n \to \infty]{} 0} \; \forall  \varepsilon > 0.$
Мы можем зафиксировать некоторый $\varepsilon_0$~--- допустимую погрешность, и найти такое $n$, что указанная вероятность будет мала~--- например, $0.01$. 
Тогда значение оценки $T_n(\SampleX)$ с вероятностью $0.99$ отклоняется от истинного значения не более чем на $\varepsilon_0$.\\
В отсутствие этого свойства статистика совершенно «несостоятельна» как оценка.


\begin{rmrk}
    Отметим некоторые свойства несмещённых и состоятельных оценок.
    \begin{enumerate}
        \item Несмещённые оценки не единственны.
        
        К примеру, в качестве несмещённой оценки для математического ожидания $\ExpTh X$ могут выступать $\ExpTh X_{1}$ или $\ExpTh \SampleMean$.
        
        \item Несмещённые оценки могут не существовать.
        \begin{exmp}
            Дано распределение $\Pois_{\theta}$, над которым произведено одно наблюдение. Найти несмещённую оценку для функции $\tau(\theta) = \cfrac{1}{\theta}$.
                \begin{gather*}
                    \ExpTh T(\SampleX) = \tau(\theta) \\
                    \mathlarger{\sum}_{x = 0}^{\infty} T(x) \, e^{-\theta} \frac{\theta^x}{x!} = \frac{1}{\theta} \\
                    \mathlarger{\sum}_{x = 0}^{\infty} T(x) \, \frac{\theta^{x + 1}}{x!} = e^{\theta}, \text{ но } e^{\theta} = \mathlarger{\sum_{k = 0}^{\infty} \frac{\theta^k}{k!}} \; \implies \\
                    T(x) \equiv \frac{1}{\theta} \quad \forall x \in \{0, 1, 2, 3, \ldots\}
                \end{gather*}
            Мы получили, что функция $T(x)$ зависит от $\theta$, но тогда она не может быть статистикой~--- ведь статистика должна зависеть только от выборки.
        \end{exmp}
        
    \item Несмещённые оценки могут существовать, но быть бессмысленными.
    \begin{exmp}
        Дано геометрическое распределение $\Geom_{1 - \theta}$ с вероятностью успеха $1 - \theta$, над которым произведено одно наблюдение.
        Найти несмещённую оценку для параметра $\theta$.
        \begin{gather*}
            \ExpTh T(\SampleX) = \theta \\
            \mathlarger{\sum}_{x = 0}^{\infty} T(x) (1 - \theta) \theta^x = \theta \\
            \mathlarger{\sum}_{x = 0}^{\infty} T(x) \theta^x = \frac{\theta}{1 -\theta} = \mathlarger{\sum}_{k = 1}^{\infty} \theta^k \; \implies \\
            T(x) = \begin{cases}
                0, & \text { если } x = 0 \\
                1, & \text { если } x \geqslant 1
            \end{cases}
        \end{gather*}
    Значения этой статистики не принадлежат параметрическому множеству $\Theta = (0, 1)$, следовательно, эта оценка бессмысленна.
    \end{exmp}
    
    \item Состоятельные оценки не единственны.
    
    Как будет показано в следующем параграфе, выборочная дисперсия $\SampleVar$ и несмещённая выборочная дисперсия $S_0^{2}$ являются состоятельными оценками теоретической дисперсии.
    
    \item Состоятельные оценки могут быть смещёнными.
    
    Например, выборочная дисперсия является состоятельной, но смещённой оценкой теоретической дисперсии.
    
    \end{enumerate}
\end{rmrk}

Как мы увидели, несмещённые и состоятельные оценки не единственны.
Возникает вопрос~--- как определить, какая из нескольких имеющихся оценок лучше?

Рассмотрим несмещённые оценки $T(\SampleX)$ параметра $\theta$, для которых существует дисперсия: $\ExpTh \bigl(T(\SampleX)-\theta\bigr)^{2}=\VarTh T(\SampleX)$.
Обозначим класс всех таких оценок $\mathcal{T}$.
Тогда можно оценивать точность оценок дисперсией.
Если $T, T^* \in \mathcal{T}$ и $\VarTh T^* \leqslant \VarTh T \quad \AllTh$, то по $T^*$ \textit{равномерно (по $\theta$) не хуже} $T$.

Введём понятие оптимальной оценки.
\begin{defn}
    \textit{Оптимальная оценка} параметра $\theta$~--- статистика $T^*(\SampleX)$, т.ч.:
    \begin{enumerate}
        \item $T^* \in \mathcal{T}$, т.е. $T^*$~--- несмещённая.
        \item $T^*$ имеет равномерно минимальную дисперсию, т.е. для любой другой \textbf{несмещённой} оценки~$T_{1} \in \mathcal{T}$ параметра~$\theta \colon$ ${\VarTh T^* \leqslant \VarTh T_{1}} \quad {\AllTh}$.
    \end{enumerate}
\end{defn}

\vspace{5mm}
\begin{thm*}
    Если существует оптимальная оценка параметра $\theta$, то она единственна.
\end{thm*}

\begin{proof}
    Предположим обратное: пусть существуют две оптимальные оценки $T_1(\SampleX)$ и $T_2(\SampleX)$ параметра $\theta$. 
    Тогда в силу их несмещённости: $\ExpTh  T_{1}(\SampleX) = \ExpTh T_{2}(\SampleX) = T(\SampleX)$,
    а в силу того, что они имеют равномерно минимальную дисперсию: $\VarTh  T_{1}(\SampleX)=\VarTh  T_{2}(\SampleX) \quad \AllTh$.

    Введём новую статистику: 
    \begin{equation*}
        T_{3}(\SampleX)=\cfrac{T_{1}(\SampleX)+T_{2}(\SampleX)}{2}
    \end{equation*}

    Так как $\ExpTh  T_{3}(\SampleX)=\cfrac{\ExpTh  T_{1}(\SampleX)+\ExpTh  T_{2}(\SampleX)}{2}=\theta$, то $T_{3}(\SampleX)$~--- несмещённая оценка параметра $\theta$.

    Имеем также:
    \begin{equation*}
        \VarTh  T_{3}(\SampleX) = 
        \cfrac{\VarTh \bigl(T_{1}(\SampleX)+T_{2}(\SampleX)\bigr)}{2^2} =
        \cfrac{\VarTh  T_{1}(\SampleX)+\VarTh  T_{2}(\SampleX)+2 \operatorname{cov}\bigl(T_{1}(\SampleX) T_{2}(\SampleX)\bigr)}{4}
    \end{equation*}

    %В силу свойства
    %\begin{equation*}
    %    \ExpTh \xi^{2}<\infty, \ExpTh \eta^{2}<\infty \implies|\operatorname{cov}(\xi, \eta)| = | \ExpTh (\xi-\ExpTh  \xi)(\eta-\ExpTh  \eta)| \leqslant \sqrt{\VarTh  \xi} \sqrt{\VarTh  \eta},
    %\end{equation*}
    По свойствам ковариции $| \operatorname{cov}(\xi, \eta)| \leqslant \sqrt{\VarTh \!\xi \, \VarTh \!\eta}$, а значит
    %где равенство достигается тогда и только тогда, когда $\xi=a \eta+b$, получаем:
    \begin{equation*}
        \VarTh  T_{3}(\SampleX) \leqslant \cfrac{\VarTh  T_{1}(\SampleX)+\VarTh  T_{2}(\SampleX)+2 \sqrt{\VarTh  T_{1}(\SampleX)} \sqrt{\VarTh  T_{2}(\SampleX)}}{4} =\VarTh  T_{1}(\SampleX)
    \end{equation*}

    В силу того, что $T_1(\SampleX)$ и $T_2(\SampleX)$ — оптимальные, дисперсия $T_3(\SampleX)$ не может быть меньше дисперсии $T_1(\SampleX)$, 
    следовательно, неравенство должно обратиться в равенство, но тогда
    \begin{equation*}
    \begin{aligned}
        T_{1}(\SampleX)=a T_{2}(\SampleX)+b \implies \ExpTh  T_{1}(\SampleX)
        = a \ExpTh  T_{2}(\SampleX)+b 
        \iff \\
        \iff \theta = a \theta + b~ \AllTh \implies a = 1, b = 0
    \end{aligned}
    \end{equation*}
\end{proof}
  % Выборочные моменты. Их свойства
\section{Выборочные моменты. Их свойства}

В параграфе 2.1 мы предположили, что все случайные величины выборки $\SampleX= \Sample$ имеют одно и то же распределение, т.е. $X_i \sim \xi~~\forall i = \overline{1, n}$ для некоторой случайной величины $\xi$. 
Попробуем найти приближения некоторых числовых характеристик этой случайной величины.
\begin{defn}
    \textit{Выборочное математическое ожидание:} 
    \begin{equation*}
        \tilde{\Exp } \xi =
        \sum\limits_{i=1}^{n} \frac{1}{n} X_{i} =
        \frac{1}{n} \sum\limits_{i=1}^{n} X_{i} =
        \SampleMean
    \end{equation*}

    Выборочное матожидание функции $g(\xi)$:
    \begin{equation*}
        \tilde{\Exp } g\left(\xi \right) = 
        \frac{1}{n} \sum\limits_{i=1}^{n} g\left(X_{i}\right) =
        \overline{g(X)}
    \end{equation*}
\end{defn}

\begin{defn}
    \textit{Выборочная дисперсия:}
    \begin{equation*}
        \tilde{\Var } \xi = 
        \sum\limits_{i=1}^{n} \frac{1}{n} \bigl(X_{i}-\tilde{\Exp } \xi \bigr)^{2} = 
        \frac{1}{n} \sum\limits_{i=1}^{n} \bigl(X_{i}-\SampleMean \bigr)^{2} =
        \SampleVar
    \end{equation*}
\end{defn}

\begin{defn}
    \textit{Несмещённая выборочная дисперсия:} 
    \begin{equation*}
        S_{0}^{2} = 
        \frac{1}{n-1} \sum\limits_{i=1}^{n}\left(X_{i}-\SampleMean\right)^{2} =
        \frac{n}{n-1} S^2
    \end{equation*}
\end{defn}

\begin{defn}
    \textit{Выборочный момент $k$-го порядка:}
    \begin{equation*}
        \tilde{\Exp } \left[ \xi^{k} \right] = 
        \sum\limits_{i=1}^{n} \frac{1}{n} X_{i}^{k} =
        \frac{1}{n} \sum\limits_{i=1}^{n} X_{i}^{k} =
        \overline{X^{k}}
    \end{equation*}
\end{defn}

Все вышеперечисленные характеристики являются случайными величинами как функции от выборки $\Sample$ и оценками для истинных моментов искомого распределения.

Введём ещё одно определение.
\begin{defn}
    Статистика $T(\SampleX)$ называется \textit{асимптотически нормальной}, если существуют такие 
    $a_n(\theta), \sigma_n(\theta)$, что $\cfrac{T_n(\SampleX) - a_n(\theta)}{\sigma_n(\theta)} \xrightarrow[n \to \infty]{\text{d}} \Normal_{0, 1}$.
    Иными словами, оценка называется асимптотически нормальной, если с ростом объёма выборки её функция распределения (оценка, будучи функцией от выборки, сама является случайной величиной) стремится к функции нормального распределения. %Забегая вперёд, отметим, что это свойство бывает полезным при построении доверительных интервалов.
\end{defn}

\begin{thm*}
    Выборочное среднее $\SampleMean$ является несмещённой, состоятельной и асимптотически нормальной оценкой для теоретического среднего (математического ожидания), то есть:

    \begin{enumerate}[label={\arabic*.}]
        \item Если $\ExpTh |X_{1}|<\infty$, то $\ExpTh \SampleMean = \ExpTh X_{1}=a$;
        \item Если $\ExpTh |X_{1}|<\infty$, то $\SampleMean \xrightarrow[n \to \infty]{\text{p}} \ExpTh X_{1}=a$;
        \item Если $\VarTh  X_{1}<\infty,~ \VarTh X_{1} \neq 0$, 
        
        то $\cfrac{\SampleMean - \ExpTh \SampleMean }{\sqrt{\VarTh \SampleMean}} = \sqrt{n} \,\cfrac{\SampleMean - \ExpTh X_1}{\sqrt{\VarTh X_1}} \xrightarrow[n \to \infty]{\text{d}} \Normal_{0, 1}$.
    \end{enumerate}
\end{thm*}

\begin{proof}
\begin{enumerate}[label={\arabic*.}]
    \item Из линейности математического ожидания:
    \begin{equation*}
        \ExpTh \SampleMean=\frac{1}{n}\bigl(\ExpTh  X_{1}+\ldots + \ExpTh X_{n}\bigr)=\frac{1}{n} \cdot n \, \ExpTh X_{1}= \ExpTh  X_{1} = a.
    \end{equation*}
    \item Из ЗБЧ в форме Хинчина:
    \begin{equation*}
        \SampleMean = 
        \cfrac{X_{1}+\ldots+X_{n}}{n} \xrightarrow[n \to \infty]{\text{p}} \ExpTh X_{1} = a.
    \end{equation*}

    \item Раскроем дисперсию суммы, пользуясь тем, что $X_1, \ldots X_n$ независимы и одинаково распределены, а затем домножим числитель и знаменатель на $n$. 
    Тогда можно будет применить ЦПТ:
    \begin{gather*}
        \frac{\bigl(\SampleMean - \ExpTh \SampleMean \bigr)}{\sqrt{\VarTh \SampleMean}} =
        \frac{\bigl(\SampleMean - \ExpTh X_{1}\bigr)}{\sqrt{\VarTh \left[ \frac{1}{n} \sum\limits_{i=1}^n X_i \right]}} = 
        \frac{\bigl(\SampleMean - \ExpTh X_{1}\bigr)}{\sqrt{\frac{1}{n^2} \VarTh \left[ \sum\limits_{i=1}^n X_i \right]}} = \\
        \frac{\bigl(\SampleMean - \ExpTh X_{1}\bigr)}{\sqrt{\frac{1}{n^2} \, n \, \VarTh X_1}} = 
        \frac{\frac{1}{n}\sum\limits_{i=1}^{n} X_{i} - \ExpTh X_{1}}{\sqrt{ \frac{1}{n} \, \VarTh X_1}} = 
        \sqrt{n} \, \frac{\frac{1}{n}\sum\limits_{i=1}^{n} X_{i} - \ExpTh X_{1}}{\sqrt{\VarTh X_1}} = \\
        \sqrt{n} \, \frac{\sum\limits_{i=1}^{n} X_{i} - n \ExpTh X_{1}}{n \sqrt{\VarTh X_1}} =
        \frac{\sum\limits_{i=1}^{n} X_{i} - n \ExpTh X_{1}}{\sqrt{n \, \VarTh X_1}} 
        \xrightarrow[n \to \infty]{\text{d}} \Normal_{0, 1} 
    \end{gather*}
\end{enumerate}
\end{proof}

\begin{rmrk}
    Аналогичными свойствами обладает выборочный $k$-й момент, являющийся несмещённой, состоятельной и асимптотически нормальной оценкой для теоретического $k$-го момента.
\end{rmrk}

\begin{rmrk}
    Применив \hyperlink{SLLN}{УЗБЧ Колмогорова}, можно показать, что выборочные $k$-е моменты сходятся к теоретическим почти наверное. 
    Такие оценки называются \textit{сильно состоятельными}. На практике обычно достаточно и состоятельности в обычном смысле (т.е. сходимости к теоретическому моменту по вероятности с ростом объёма выборки).
\end{rmrk}

\begin{thm*}
    Пусть $\VarTh X_{1}<\infty$.
    \begin{enumerate}
        \item Выборочные дисперсии $\SampleVar$ и $\SampleVar_0$ являются состоятельными оценками для истинной дисперсии:
            \begin{equation*}
                \SampleVar \xrightarrow[n \to \infty]{\text{p}} \VarTh X_{1}=\sigma^{2}, \quad S_{0}^{2} \xrightarrow[n \to \infty]{\text{p}} \VarTh X_{1}=\sigma^{2} \quad \AllTh.
            \end{equation*}
        \item $\SampleVar$~--- смещённая оценка дисперсии, а $\SampleVar_0$~--— несмещённая:
            \begin{equation*}
                \ExpTh \SampleVar=\frac{n-1}{n} \, \VarTh X_{1}=\frac{n-1}{n} \sigma^{2} \neq \sigma^{2}, \quad \ExpTh  S_{0}^{2}=\VarTh  X_{1}=\sigma^{2} \quad \AllTh.
            \end{equation*}
        \item Если $0 \neq \VarTh \Bigl[ \bigl(X_{1}-\ExpTh X_{1}\bigr)^{2} \Bigr] <\infty$, то $\SampleVar$ и $\SampleVar_0$ являются асимптотически нормальными оценками истинной дисперсии:
            \begin{equation*}
                \sqrt{n}\left(\SampleVar-\VarTh  X_{1}\right) \xrightarrow[n \to \infty]{\text{d}} \Normal_{0, \VarTh \bigl(X_{1}-\ExpTh  X_{1}\bigr)^{2}} \quad \AllTh.
            \end{equation*}
    \end{enumerate}
\end{thm*}

\begin{proof}
\begin{enumerate}
    \item 
    $\SampleVar = 
    \cfrac{1}{n} \sum\limits_{i=1}^{n}\left(X_{i}-\SampleMean\right)^{2} = 
    \cfrac{1}{n} \sum\limits_{i=1}^{n}\Bigl(X_i^2 - 2 X_i \SampleMean + \bigl(\SampleMean\bigr)^2 \Bigr) = \\
    \cfrac{1}{n} \Bigl( n \overline{X^2} - 2 \SampleMean \sum\limits_{i=1}^{n}X_i + n \bigl(\SampleMean\bigr)^2\Bigr) = 
    \overline{X^{2}}-2\bigl(\SampleMean\bigr)^{2} + \bigl(\SampleMean\bigr)^2 = 
    \overline{X^{2}}-\bigl(\SampleMean\bigr)^{2}$.

    Используя состоятельность первого и второго выборочных моментов и свойства сходимости по вероятности, получаем:
    \begin{gather*}
        \SampleVar=\overline{X^{2}}-(\SampleMean)^{2} \xrightarrow[n \to \infty]{\text{p}} \ExpTh  X_{1}^{2} - \bigl(\ExpTh  X_{1}\bigr)^{2}=\sigma^{2} \\
        \cfrac{n}{n-1} \underset{n \to \infty}{\longrightarrow} 1 \implies S_{0}^{2}=\frac{n}{n-1} \SampleVar \xrightarrow[n \to \infty]{\text{p}} \sigma^{2}
    \end{gather*}
    
    \item Используя несмещённость первого и второго выборочных моментов:
    \begin{multline*}
        \ExpTh  \SampleVar = \ExpTh \left(\overline{X^{2}}-(\SampleMean)^{2}\right)
        = \ExpTh \overline{X^{2}}-\ExpTh \bigl(\SampleMean\bigr)^{2}
        = \ExpTh X_{1}^{2}-\ExpTh \bigl(\SampleMean\bigr)^{2} = \\
        = \ExpTh X_{1}^{2}-\Bigl(\bigl(\ExpTh \SampleMean\bigr)^{2} + \VarTh \SampleMean\Bigr)
        = \ExpTh X_{1}^{2}-\bigl(\ExpTh X_{1}\bigr)^{2} - \VarTh \left(\frac{1}{n} \sum\limits_{i=1}^{n} X_{i}\right) = \\
        = \VarTh X_1 - \VarTh \left(\frac{1}{n} \sum\limits_{i=1}^{n} X_{i}\right)
        = \sigma^{2}-\frac{1}{n^{2}} n \, \VarTh  X_{1}
        = \sigma^{2}-\frac{\sigma^{2}}{n}
        = \frac{n-1}{n} \sigma^{2}.
    \end{multline*}
    
    Откуда следует:
    \begin{equation*}
        \ExpTh S_{0}^{2}=\frac{n}{n-1} \, \ExpTh \SampleVar=\sigma^{2}.
    \end{equation*}
    
    \item Введём случайные величины $Y_{i}=X_{i}-a$; $\ExpTh Y_{i} = 0, \; \VarTh Y_{1} = \VarTh X_{1}=\sigma^{2}$.
    \begin{gather*}
        \SampleVar=\frac{1}{n} \sum\limits_{i=1}^{n}(X_{i}-\SampleMean)^{2}=\frac{1}{n} \sum\limits_{i=1}^{n}(X_{i}-a-(\SampleMean-a))^{2}=\overline{Y^{2}}-\bigl(\overline{Y}\bigr)^{2}. \\
        \begin{aligned}
            \sqrt{n}\bigl(\SampleVar-\sigma^{2}\bigr) = \sqrt{n}\Bigl(\overline{Y^{2}}-\bigl(\overline{Y}\bigr)^{2}-\sigma^{2}\Bigr)
            = \sqrt{n}\Bigl(\overline{Y^{2}}-\ExpTh  Y_{1}^{2} \Bigr) - \sqrt{n}\bigl(\overline{Y}\bigr)^{2} = \\
            =\frac{\sum\limits_{i=1}^{n} Y_{i}^{2}-n \ExpTh  Y_{1}^{2}}{\sqrt{n}}-\overline{Y} \cdot \sqrt{n} \, \overline{Y} \xrightarrow[n \to \infty]{\text{d}} \Normal_{0, \VarTh (X_{1}-\ExpTh X_1}^{2}),
    \end{aligned}
    \end{gather*}
    поскольку $\cfrac{\sum\limits_{i=1}^{n} Y_{i}^{2}-n \ExpTh  Y_{1}^{2}}{\sqrt{n}} \xrightarrow[n \to \infty]{\text{d}} \Normal_{0, \VarTh  Y_{1}^{2}}$ по ЦПТ, 
    а ${\overline{Y} \cdot \sqrt{n} \, \overline{Y} \xrightarrow[n \to \infty]{\text{d}} 0}$ как произведение последовательностей ${\overline{Y} \xrightarrow[n \to \infty]{p} 0}$ и ${\sqrt{n} \, \overline{Y} \xrightarrow[n \to \infty]{\text{d}} \Normal_{0, \VarTh  X_{1}}}$.
\end{enumerate}
\end{proof}

  % Точечная оценка. Несмещённость, состоятельность, оптимальность. Теорема о единственности оптимальной оценки
\section{Функция правдоподобия. Достаточные статистики, полные статистики. Теорема факторизации}

В зависимости от типа семейства распределения $\Dist_\theta$ обозначим через $f_{\theta}(y)$ одну из следующих функций:
\begin{equation*}
    f_{\theta}(y) =
    \left\{\begin{array}{ll}
    \text { плотность } f_{\theta}(y), & \text { если } \Dist_{\theta} \text { абсолютно непрерывно, } \\
    P_{\theta}\left(X_{1}=y\right), & \text { если } \Dist_{\theta} \text { дискретно. }
    \end{array}\right.
\end{equation*}

\begin{defn}
    \textit{Функция правдоподобия} выборки $\SampleX$:
    \begin{equation*}
        L(\SampleX, \theta)=f_{\theta}\left(X_{1}\right) \cdot f_{\theta}\left(X_{2}\right) \cdot \ldots \cdot f_{\theta}\left(X_{n}\right)=\prod_{i=1}^{n} f_{\theta}\left(X_{i}\right).
    \end{equation*}
\end{defn}

В дискретном случае функция правдоподобия принимает вид:
\begin{equation*}
    \begin{aligned}
        L(\SampleX, \theta)=\prod_{i=1}^{n} f_{\theta}(x_{i}) 
        = \MyPrTh (X_{1}=x_{1}) \cdot \ldots \cdot \MyPrTh (X_{n}=x_{n}) = \\
        = \MyPrTh (X_{1}=x_{1}, \ldots, X_{n}=x_{n}).
    \end{aligned}
\end{equation*}

\begin{rmrk}
    Функция правдоподобия зависит от выборки~$\SampleX$ и от параметра~$\theta$.
    Если мы зафиксируем некоторое~$\theta_0$, то получим вероятностную меру над выборочным пространством~$\Real^n$.
    Тогда значение функции правдоподобия~$L(\mathbf{x}, \theta_0)$ на некоторой реализации выборки~$x$~--- это вероятность этой реализации в предположении, что параметр~$\theta_0$~--- истинное значение. \\
    Если же мы напротив, зафиксируем некоторую реализацию выборки~${\mathbf{x}^* = (x_1^*, \ldots, x_n^*)}$ (что соответствует реальным условиям~--- как правило, у нас есть некоторые наблюдения, а параметр неизвестен), то значение функции правдоподобия на некотором параметре~$\theta$~--- это "<правдоподобие">, "<вероятность"> этого параметра.
    Но, строго говоря, при фиксированной реализации выборки~$x^*$ функция правдоподобия~$L(\mathbf{x}^*, \theta)$ не является вероятностной мерой над~$\Theta$.
    Рассмотрим пример.
\end{rmrk}

\begin{exmp}
    При двух последовательных независимых подбрасываниях монетки выпало два "<орла">.
    Запишем функцию правдоподобия выборки $\SampleX= \bigl(X_1, X_2\bigr), \; X_i \in \{0, 1\}$.
    Обозначим вероятность выпадения "<орла"> ${\bigl(\MyPrTh(X_i = 1)\bigr)}$ за $\theta, \; \theta \in [0, 1]$.
    Тогда
    \begin{equation*}
        L(\SampleX, \theta) = \MyPrTh (X_1 = x_1) \cdot \MyPrTh (X_2 = x_2).
    \end{equation*}
    Для любого фиксированнного значения параметра $\theta_0$ функция правдоподобия $L(\SampleX, \theta_0)$ является функцией совместного распределения $(X_1, X_2)$:
    \begin{center}
        \begin{tabular}{|c|c|c|}
            \hline
            & $X_1 = 1$ & $X_1 = 0$ \\
            \hline
            $X_2 = 1$ & $\theta^2_0$ & $(1-\theta_0)\theta_0$ \\
            \hline
            $X_2 = 0$ & $\theta_0(1-\theta_0)$ & $(1-\theta_0)^2$ \\
            \hline
        \end{tabular}
    \end{center}
    \begin{gather*}
        \sum\limits_{x_1 = 0}^{1} \sum\limits_{x_2 = 0}^{1} L(\mathbf{x}, \theta_0) = 
        \sum\limits_{x_1 = 0}^{1} \sum\limits_{x_2 = 0}^{1} \MyPr{\theta_0} (X_1 = x_1) \cdot \MyPr{\theta_0} (X_2 = x_2) = \\
        = \theta_0^2 + (1 - \theta_0)\theta_0 + \theta_0(1-\theta_0) + (1 - \theta_0)^2 
        = \theta_0^2 + 2\theta_0 - 2\theta_0^2 + (1 - 2\theta_0 + \theta_0^2)
        = 1.
    \end{gather*}
    Но если мы зафиксируем реализацию выборки $x^* = (x_{1}^{*} = 1, x_{2}^{*} = 1)$ и проинтегрируем по $\Theta = [0, 1]$, то получим
    \begin{equation*}
        \int\limits_0^1 L(\mathbf{x}^*, \theta) \, d\theta = 
        \int\limits_0^1 \MyPrTh(X_1 = 1) \cdot \MyPrTh(X_2 = 1) \, d\theta = 
        \int\limits_0^1 \theta^2 \, d\theta = 
        \left. \frac{\theta^3}{3} \right|_0^1 = 
        \frac{1}{3}.
    \end{equation*}

    Выходит, что $L(\mathbf{x}^*, \theta)$ не может быть вероятностной мерой, так как интеграл по $\Theta$ не равен 1.
    %А $L(\SampleX, \theta_0)$ определяет совместное распределение случайных величин $(X_1, X_2)$: если они абсолютно непрерывны~--- то это совместная плотность, а если дискретны~--- то вероятность принятия соответствующих значений.
\end{exmp}

%Таким образом, смысл функции правдоподобия~--- <<вероятность>>\footnote{Строго говоря, функция правдоподобия не является вероятностной мерой над $\theta$, хотя бы потому, что $\int\limits_{\Theta} L(\mathbf{x}, \theta) d\theta \neq 1$. 
%Термин <<вероятность>> применён здесь в переносном смысле.} попасть в заданную точку при соответствующем параметре $\theta$ в дискретном случае; для абсолютно непрерывного аналогично~--- вероятность попасть в куб с центром в $x_1, \ldots, x_n$ и сторонами $dx_1, \ldots, dx_n$.

\vspace{5mm}
\begin{defn}
    \textit{Достаточная статистика}~--- статистика $T(\SampleX)$ такая, что $\forall t \in \Real^m,$~\footnote{Напомним, что вообще говоря, статистика~--- 
    это отображение $T(X)\colon \Real^n \mapsto \Real^m$ } $ \forall B \in \Borel^n$ условное распределение $\MyPrTh \Bigl(\Sample \in B | \, T=t\Bigr)$ не зависит от параметра $\theta$. Иными словами, статистика $T(\SampleX)$ называется достаточной, если $\ExpTh \bigl( \SampleX\, | \, T(\SampleX) \bigr)$ не зависит от $\theta$.
\end{defn}

Иными словами, достаточная статистика содержит в себе всю информацию о параметре, которую можно извлечь из выборки.

Достаточная статистика всегда существует~--- например, сама выборка $\SampleX$ является достаточной статистикой.
Однако этот тривиальный пример не очень полезен.
Как правило, нас интересуют достаточные статистики малой размерности.
Для того, чтобы проверить, является ли статистика достаточной, можно использовать критерий факторизации.
\begin{namedthm}[Критерий факторизации]
    $T(\SampleX)$~--- достаточная статистика $\iff$ её функция правдоподобия представима в виде 
    \begin{equation*}
        L(X_{1}, \ldots, X_{n} , \theta) \stackrel{\text{п.н.}}{=} g\bigl(T(\SampleX), \theta\bigr) \cdot h(\SampleX)
    \end{equation*}
\end{namedthm}

\begin{proof}
    Рассмотрим только дискретный случай. 
    \begin{enumerate}
        \item[$\implies$] Пусть $T(\SampleX)$~--- достаточная статистика. 
            Возьмём произвольную реализацию выборки $\boldsymbol{x}$ и обозначим $t = T(\boldsymbol{x})$. Тогда
            \begin{multline*}
                L(\boldsymbol{x}, \theta) = \MyPrTh (\SampleX=\boldsymbol{x})=\MyPrTh (\SampleX=\boldsymbol{x}, T(\SampleX)=t) =\\
                = \underbrace{\MyPrTh (T(\SampleX)=t)}_{g(T(\SampleX), \theta)} \underbrace{\MyPrTh (\SampleX=\boldsymbol{x} | T(\SampleX)=t)}_{h(\SampleX)}
            \end{multline*}
        \item[$\impliedby$] Пусть теперь функция правдоподобия имеет вид $L(\boldsymbol{x}, \theta)=g\bigl(T(\boldsymbol{x}), \theta\bigr) h(\boldsymbol{x})$. 
            Тогда, если $x$ таково, что $T(\boldsymbol{x})=t$, то:
            \begin{multline*}
                \MyPrTh (\SampleX=\boldsymbol{x} | \, T(\SampleX)=t) =\frac{\MyPrTh (\SampleX=\boldsymbol{x}, T(\SampleX)=t)}{\MyPrTh (T(\SampleX)=t)}
                =\frac{\MyPrTh (\SampleX=\boldsymbol{x})}{\sum\limits_{\boldsymbol{x}^{\prime}: T(\boldsymbol{x}^{\prime})=t} \MyPrTh(\SampleX=\boldsymbol{x}^{\prime})} = \\
                = \frac{g(t, \theta) h(\boldsymbol{x})}{\sum\limits_{\boldsymbol{x}^{\prime}: T(\boldsymbol{x}^{\prime})=t} g(t, \theta) h(\boldsymbol{x}^{\prime})}
                = \frac{h(\boldsymbol{x})}{\sum\limits_{\boldsymbol{x}^{\prime}: T(\boldsymbol{x}^{\prime})=t} h(\boldsymbol{x}^{\prime})}
            \end{multline*}
            Т.е. вероятность не зависит от $\theta$, а значит, $T(\SampleX)$ достаточная.
    \end{enumerate}
\end{proof}

\begin{defn}
    Статистика $T(\SampleX)$ называется \textit{полной}, если для любой борелевской функции $\varphi(T)$, для которой $\ExpTh \varphi(T)=0~\AllTh$, справедливо: $\varphi(T) \stackrel{\text{п.н.}}{=}0$.
\end{defn}

\begin{exmp}
    Рассмотрим равномерное распределение $\Uniform_{[0,\theta]}$ и докажем, что статистика $T(\SampleX) = X_{(n)}$ является полной. 
    Напоминаем, что плотность распредения $X_{(n)}$ для $U[0, \theta]$ - это $f(t) = \cfrac{n t^{n-1}}{\theta^n} \, \Ind(0 \leqslant t \leqslant \theta)$.
    \begin{gather*}
        \ExpTh \varphi(X_{(n)})= \int\limits_{0}^{\theta} \varphi(t) \cfrac{nt^{n-1}}{\theta^n} dt = 0 \implies \int\limits_{0}^{\theta} \varphi(t) t^{n-1} dt = 0 \implies \\
        \{ \text{продифференцируем по}~ \theta\} \implies 
        \varphi(\theta) \theta^{n-1} = 0  \implies \varphi(T) \equiv 0 ~~ \forall T > 0
    \end{gather*}
    Таким образом, $\varphi(T) \stackrel{\text{п.н.}}{=} 0$ при $T \geqslant 0$, и указанная статистика является полной.
\end{exmp}
  % Функция правдоподобия. Достаточные статистики, полные статистики. Теорема факторизации
\section{Неравенство Рао"--~Крамера. Эффективные оценки}
Пусть $X_1, \ldots, X_n$  —  некоторая выборка с функцией правдоподобия $L(\mathbf{X}, \theta)$ относительно некоторой меры $\mu$. Введём функцию ${\varphi(\theta)=\int\limits_{\mathbf{R}^{n}} T(x) L(\mathbf{x}, \theta) \mu(d x)<\infty}$, в дальнейшем считая, что она дифференцируема необходимое число раз.

\begin{defn}
    Функция правдоподобия $L(\SampleX, \theta)$ \textit{удовлетворяет условиям регулярности для $m$-й производной}, если существует
    \begin{equation*}
        \cfrac{d^{m} \varphi(\theta)}{d \theta^{m}}=\int\limits_{\Real^{n}} T(x) \cfrac{\partial^{m}}{\partial \theta^{m}}L(\mathbf{x}, \theta) \mu(d x),
    \end{equation*}
    причём множество $\left\{ {x\colon L(\mathbf{x},\theta) > 0} \right\}$ не зависит от параметра $\theta$.
\end{defn}

\begin{exmp}
    Условиям регулярности удовлетворяют многие модели: биномиальная, пуассоновская, нормальная, гамма-распределение (а следовательно, и экспоненциальная, и $\chi^2$) и т.д.
    Требование независимости множества значений исследуемой случайной величины от параметра $\theta$ существенно: например, равномерное распределение $\Uniform_{[0, \theta]}$ относится к нерегулярным моделям.
    Это происходит из-за того, что при дифференцировании интеграла, пределы интегрирования которого зависят от параметра, появляются дополнительные слагаемые. 
    Т.е. менять местами интегрирование и дифференцирование нельзя, и к нерегулярным моделям изложенная ниже теория неприменима.
\end{exmp}

\begin{defn}
    Функция $U(\SampleX, \theta) = \cfrac{\partial \ln L(\SampleX, \theta)}{\partial \theta}$ называется \textit{функцией вклада}.
\end{defn}
\textbf{Утверждение.} 
    Если функция правдоподобия удовлетворяет условиям регулярности для первой производной, то $\ExpTh U(\SampleX, \theta) = 0$. 
    В самом деле,
    \begin{multline*}
        \ExpTh U(\SampleX, \theta) = \int\limits_{\Real^n} U(x, \theta) L(\mathbf{x}, \theta) \mu(dx) = 
        \int\limits_{\Real^n} \cfrac{\partial \ln L(\mathbf{x}, \theta)}{\partial \theta} L(\mathbf{x}, \theta) \mu(dx) = \\
        \int\limits_{\Real^n} \cfrac{1}{L(\mathbf{x}, \theta)} \cfrac{\partial L(\mathbf{x}, \theta)}{\partial \theta} L(\mathbf{x}, \theta) \mu(dx) = 
        \int\limits_{\Real^n} \cfrac{\partial L(\mathbf{x}, \theta)}{\partial \theta} \mu(dx) = \{\text{регулярность}\}\\
        = \cfrac{\partial}{\partial \theta} \int\limits_{\Real^n} L(\mathbf{x}, \theta) \mu(dx) 
        = \cfrac{\partial}{\partial \theta} ~ 1 = 0.
    \end{multline*}
Из этого, в частности, вытекает, что для регулярных моделей ${\VarTh U(\SampleX, \theta) = \ExpTh U^2(\SampleX, \theta)}$.

\vspace{5mm}
Посчитаем дисперсию функции вклада:
\begin{multline*}
    \VarTh U(\SampleX, \theta) = 
    \VarTh \cfrac{\partial \ln L(\SampleX, \theta)}{\partial \theta} = 
    \VarTh \cfrac{\partial \ln\prod\limits_{i = 1}^n { f_{\theta}(X_i)} }{\partial \theta}  = \\
    \VarTh \cfrac{\partial \sum\limits_{i = 1}^n {\ln f_{\theta}(X_i)} }{\partial \theta}  = 
    \VarTh \sum\limits_{i = 1}^n \cfrac{\partial {\ln f_{\theta}(X_i)} }{\partial \theta}  =
    \{\text{независимость выборки}\} = \\
    \sum\limits_{i = 1}^n \VarTh \cfrac{\partial {\ln f_{\theta}(X_i)} }{\partial \theta}  = 
    \{\text{однородность выборки}\} = \\
    n \,\VarTh \cfrac{\partial {\ln \,f_{\theta}(X_1)} }{\partial \theta}  = 
    n \,\VarTh U(X_1, \theta) = n \, i_1(\theta).
\end{multline*}
Здесь за $i_1(\theta)$ обозначена дисперсия функции вклада от выборки из одного элемента.
\hypertarget{fisher}{}
\begin{defn}
    Пусть $\SampleX= \Sample$~--- выборка объёма $n$.
    Величину $i_n(\theta) = \VarTh U(\SampleX, \theta)$ называют \textit{фишеровской информацией, содержащейся в выборке размера $n$}.
\end{defn}

\begin{namedthm}[Неравенство Рао"--~Крамера]
    Пусть $\Sample$ — выборка, $L(\SampleX, \theta)$ удовлетворяет условиям регулярности для первой производной и $\tau(\theta)$  —  дифференцируемая функция $\theta$. Тогда:
    \begin{enumerate}
        \item Для любой $~T(\SampleX)$~--- несмещённой оценки $\tau(\theta)$, справедливо неравенство:
        \begin{gather*}
            \VarTh T(\SampleX) \geqslant 
            \cfrac{\bigl( \tau'(\theta) \bigr)^2}{n \, i_1(\theta)} = 
            \cfrac{\bigl( \tau'(\theta) \bigr)^2}{\VarTh U(\SampleX, \theta)}
            \quad \AllTh
        \end{gather*}
        
        \item Равенство достигается $\iff \; \exists a_n(\theta)\colon ~ T(\SampleX)-\tau(\theta)=a_{n}(\theta) \cdot U(\SampleX, \theta)$
    \end{enumerate}
\end{namedthm}

\begin{proof}
    Из условий регулярности $L(\SampleX, \theta)$ для следует:
    \begin{gather*}
        \int L(\mathbf{x}, \theta) \mu(d x)=1 
        \quad \implies \quad
        \int \cfrac{\partial L(\mathbf{x}, \theta)}{\partial \theta} \mu(d x)=0 \\
        \int T(x) L(\mathbf{x}, \theta) \mu(d x)=\ExpTh T(\SampleX)=\tau(\theta) 
        \quad \implies \quad
        \int T(x) \cfrac{\partial L(\mathbf{x}, \theta)}{\partial \theta} \mu(d x)=\tau'(\theta)
    \end{gather*}

    Заметим, что
    \begin{equation*}
        \cfrac{\partial L(\mathbf{x}, \theta)}{\partial \theta}=\cfrac{\partial \ln L(\mathbf{x}, \theta)}{\partial \theta} \cdot L(\mathbf{x}, \theta)
    \end{equation*}

    Откуда следует:
    \begin{gather*}
        \int U(x, \theta) L(\mathbf{x}, \theta) \mu(d x) = 0 \; \iff \; \ExpTh \bigl[U(\SampleX, \theta)\bigr]=0 \\
        \int T(x) U(x, \theta) L(\mathbf{x}, \theta) \mu(d x) = \tau'(\theta) \; \iff \; \ExpTh \bigl[ T(\SampleX) U(\SampleX, \theta) \bigr] = \tau'(\theta)
    \end{gather*}

    Вычитая из второго равенства первое, умноженного на $\tau(\theta) = \ExpTh \left[ T(\SampleX) \right]$, получаем:
    \begin{equation*}
        \ExpTh\left[T(\SampleX) U(\SampleX, \theta)\right] - \ExpTh\left[T(\SampleX)\right] \, \ExpTh\left[U(\SampleX, \theta)\right] = 
        \tau'(\theta) - 0 \cdot \tau(\theta) = 
        \tau'(\theta)
    \end{equation*}

    В левой части полученного равенства стоит ковариация случайных величин $T(\SampleX)$ и $U(\SampleX,\theta)$:
    \begin{equation*}
        \operatorname{cov}_{\theta}\bigl(T(\SampleX), U(\SampleX, \theta)\bigr)=\tau'(\theta)
    \end{equation*}

    Из неравенства Коши-Буняковского:
    \begin{equation*}
        \bigl(\tau'(\theta)\bigr)^{2}=\operatorname{cov}_{\theta}^{2} \bigl(T(\SampleX), \, U(\SampleX, \theta)\bigr) \leqslant 
        \VarTh T(\SampleX) \,\VarTh U(\SampleX, \theta) = 
        \VarTh T(\SampleX) \, n \, i_1(\theta),
    \end{equation*}

    что равносильно п.1 теоремы:
    \begin{equation*}
        \VarTh T(\SampleX) \geqslant \cfrac{\bigl(\tau'(\theta)\bigr)^{2}}{n \, i_1(\theta)}
    \end{equation*}

    Равенство достигается, если статистика и функция вклада линейно связаны (опять же, следствие неравенства Коши-Буняковского):
    \begin{equation}
        \label{connection_of_efficient_estimator_and_score}
        T(\SampleX)=\varphi(\theta) U(\SampleX, \theta)+\psi(\theta) 
        \; \implies \;
        \tau(\theta)=\psi(\theta), \; a_{n}(\theta)=\varphi(\theta).
    \end{equation}
\end{proof}

\iffalse
% Определение Черновой, конфликтующее с обозначениями в нашем курсе
    Рассмотрим некоторый класс оценок $K=\left\{\hat{\theta}\left(\SampleX\right)\right\}$ параметра $\theta$.
    \begin{defn}
        Говорят, что оценка $\theta^{*}\left(\SampleX\right) \in K$ является эффективной оценкой параметра $\theta$ в классе $K$, если для любой другой оценки $\hat{\theta} \in K$ имеет место неравенство:
        \begin{equation*}
            E\left(\theta^{*}-\theta\right)^{2} \leqslant \ExpTh(\hat{\theta}-\theta)^{2}~ \forall \theta \in \Theta
        \end{equation*}
    \end{defn}
    Обозначим класс несмещённых оценок:
    \begin{equation*}
        K_{0}=\left\{\hat{\theta}\left(\SampleX\right): E \hat{\theta}=\theta, \forall \theta \in \Theta\right\}
    \end{equation*}
    Оценка, эффективная в $K_0$ называется просто \textit{эффективной}.

    Для оценки $\theta^{*} \in K_{0}$ по определению дисперсии
    \begin{equation*}
        \ExpTh\left(\theta^{*}-\theta\right)^{2}=\ExpTh\left(\theta^{*}-\ExpTh \theta^{*}\right)^{2}=\VarTh \theta^{*}
    \end{equation*}
\fi

Оценки, обращающие неравенство Рао"--~Крамера в равенство, выделяются в отдельный класс.
\begin{defn}
    \textit{Эффективная оценка} $T(\SampleX)$~--- это несмещённая оценка параметра $\theta$ (или функции $\tau(\theta)$), дисперсия которой совпадает с нижней гранью в неравенстве Рао"--~Крамера.
\end{defn}

\begin{rmrk}
    Если существует эффективная оценка для функции $\tau(\theta)$, то ни для какой другой функции от $\theta$, кроме линейного преобразования $\tau(\theta)$, эффективной оценки существовать не будет. 
    Это следует из \eqref{connection_of_efficient_estimator_and_score}.
\end{rmrk}
Так как дисперсия любой оценки в регулярной модели не может быть меньше нижней грани, определяемой неравенством Рао"--~Крамера, то каждая эффективная оценка является оптимальной.
Обратное, в силу предыдущего замечания, неверно.
  % Неравенство Рао—Крамера. Эффективные оценки
\section{Теорема Рао"--~Блекуэлла"--~Колмогорова. Оптимальность оценок, являющихся функцией полной достаточной статистики}

\begin{namedthm}[Теорема Рао"--~Блекуэлла"--~Колмогорова] 
    Если существует оптимальная оценка для $\tau(\theta)$, то она является функцией от достаточной статистики.
\end{namedthm}

\begin{proof}
    В доказательстве используются следующие свойства условного математического ожидания: 
    \begin{gather*}
        \Exp f(x, z)=\Exp \Bigl(\Exp \bigl( f(x, z) | z\bigr) \Bigr), \\
        \Exp \bigl(g(z) | z\bigr) = g(z).
    \end{gather*}

    Мы докажем даже более сильное утверждение: для любой несмещённой оценки мы можем построить новую оценку, являющуюся функцией от достаточной статистики, при этом дисперсия построенной оценки будет не больше исходной.
    Отсюда вытекает и утверждение теоремы~--- ведь оптимальная оценка является несмещённой, соответственно, мы можем построить новую оценку, которая будет равномерно по $\theta$ не хуже оптимальной.
    Но оптимальная оценка единственна, а значит, она сама является функцией от достаточной статистики, что и требуется доказать.

    \begin{enumerate}
        \item 
            Построим искомую оценку.
            Пусть $T(\SampleX)$~--- достаточная статистика, $T_1(\SampleX)$~--- несмещённая оценка $\tau(\theta)$, т.е. $\ExpTh T_{1}(\SampleX)=\tau(\theta)$. 
            Рассмотрим функцию $H(T)=\ExpTh\left(T_{1} | T\right)$. 
            Тогда из первого свойства следует:
            \begin{multline*}
                \ExpTh H(T)=\ExpTh\bigl(\ExpTh\left(T_{1} | T\right)\bigr) = 
                \ExpTh T_{1}=\tau(\theta) \implies \\ H(T) \text{~--- несмещённая оценка~} \tau(\theta).
            \end{multline*}

        \item 
            Покажем, что её дисперсия не превосходит дисперсию исходной:
            \begin{gather*}
                \VarTh T_1 = \ExpTh \bigl(T_1 - \tau(\theta)\bigr)^2 = \\
                \ExpTh \Bigl(T_1 - H(T) + H(T) - \tau(\theta) \Bigr)^2 = \\
                \underbrace{\ExpTh \bigl(T_1 - H(T) \bigr)^2}_{\geqslant 0} + \underbrace{2 \, \ExpTh \Bigl[\bigl(T_1 - H(T)\bigr) \bigl(H(T) - \tau(\theta)\bigr)\Bigr]}_{?} + \, \VarTh H(T)
            \end{gather*}
            
            Оценим второе слагаемое, пользуясь тем, что $H(T)$~--- функция от $T$ и может быть вынесена из под условного математического ожидания как константа:
            \begin{gather*}
                \ExpTh \Bigl[ \bigl((T_{1}-H(T)\bigr) \bigl( H(T) - \tau(\theta) \bigr) \Bigr] = \\
                = \ExpTh\biggl( \ExpTh\Bigl[ \bigl(T_{1}-H(T)\bigr) \bigl( H(T)-\tau(\theta) \bigr) | \, T \Bigr] \biggr) = \\
                = \ExpTh\biggl( \bigl( H(T)-\tau(\theta) \bigr) \, \ExpTh \Bigl[ \bigl(T_{1} - H(T)\bigr) | \, T \Bigr] \biggr) = \\
                = \ExpTh\biggl( \bigl( H(T)-\tau(\theta) \bigr) \Bigl[ \ExpTh \bigl(T_{1} | T\bigr) - \ExpTh \bigl(H(T) | \, T \bigr) \Bigr] \biggr) = \\
                = \ExpTh\biggl( \bigl( H(T)-\tau(\theta) \bigr) \Bigl[ H(T) - H(T) \Bigr] \biggr) = 0.
            \end{gather*}

            Отсюда и вытекает, что
            \begin{equation*}
                \VarTh T_{1} = \underbrace{\ExpTh \bigl(T_1 - H(T)\bigr)^2}_{\geqslant 0} + \VarTh H(T) \; \geqslant \; \VarTh H(T).
            \end{equation*}
    \end{enumerate}
    Таким образом, если существует оптимальная оценка $T_1$, то $H(T)$ тоже оптимальна, но мы знаем, что оптимальная оценка единственна.
    Осталось заметить, что $H(T) = \ExpTh \bigl( T_1 | T\bigr) \equiv f(T)$~--- функция от достаточной статистики.
\end{proof}

\begin{namedthm}[Теорема Колмогорова]
    %Если $T(\SampleX)$~--- полная достаточная статистика, то она является оптимальной оценкой своего математического ожидания. (Можно сформулировать более сильный вариант)
    Пусть $T(\SampleX)$~--- полная достаточная статистика.
    Пусть $\varphi(x)$~--- борелевская функция, и $\ExpTh \varphi\bigl(T(\SampleX)\bigr) = \tau(\theta)$.
    Тогда $\varphi\bigl(T(\SampleX)\bigr)$~--- оптимальная оценка для $\tau(\theta)$.
\end{namedthm}

\begin{proof}
    Из теоремы Рао"--~Блекуэлла"--~Колмогорова следует, что если существует несмещённая оценка для $\tau(\theta)$, 
    то существует также и несмещённая оценка $\tau(\theta)$, являющаяся функцией от достаточной статистики.

    Пусть $T_1 = \varphi_1(T)$~--- несмещённая оценка $\tau(\theta)$ и $T$~--- полная достаточная статистика.
    Допустим, что существует ещё одна несмещённая оценка $T_2 = \varphi_2(T)$.
    Тогда $\AllTh$
    \begin{equation*}
        \ExpTh \Bigl(\varphi_1\bigl(T(\SampleX)\bigr) - \varphi_2\bigl(T(\SampleX)\bigr) \Bigr) = 0.
    \end{equation*}
    Но $T(\SampleX)$ полная, $\implies \; \varphi_1(y) \overset{\text{п.н.}}{=} \varphi_2(y)$ на области значений $T(\SampleX)$.
    Таким образом, несмещённая оценка, являющаяся функцией от достаточной статистики, единственна.
    Значит, она и будет оптимальной оценкой $\tau(\theta)$.

    %Докажем, что $T(\SampleX)$ является единственной несмещённой оценкой для $\ExpTh T(\SampleX)$. (Несмещённых оценок много, единственна несмещённая оценка--функция от ПДС)
    %Тогда $T(\SampleX)$ будет оптимальной оценкой. 
    %Предположим, что $T_1(\SampleX)$~--- оптимальная оценка для $\ExpTh T(\SampleX)$. 
    %Из теоремы Рао"--~Блекуэлла"--~Колмогорова получаем, что $T_{1}=H(T)$ и $\ExpTh T_{1}=\ExpTh T$. 
    %Тогда:
    %\begin{equation*}
    %    \ExpTh \underbrace{\Bigl(T(\SampleX)-H\bigl(T(\SampleX)\bigr)\Bigr)}_{\varphi(T)}=0
    %\end{equation*}
    %
    %Из условия полноты $T(\SampleX)$ следует, что $\varphi(T)=0$ с вероятностью 1, т.е. $T=H(T)$ с вероятностью 1.
\end{proof}
  % Теорема Рао—Блекуэлла—Колмогорова. Оптимальность оценок, являющихся функцией полной достаточной статистики
\section{Метод моментов. Свойства оценок, полученных методом моментов}

Пусть $\Sample$~--- выборка объёма $n$ из параметрического семейства распределений $\Dist_\theta$. 
Выберем функцию $g(y)\colon \Real \mapsto \Real$ так, чтобы существовал момент $\ExpTh g\left(X_{1}\right)=h(\theta)$ и функция $h(\theta)$ была обратима на $\Theta$. 
Разрешим полученное уравнение относительно $\theta$, а затем вместо истинного момента возьмём выборочный:

\begin{equation*}
    \theta = h^{-1}\bigl(\ExpTh g\left(X_{1}\right)\bigr), \quad 
    \theta^{*}=h^{-1}\bigl(\overline{g(\SampleX)}\bigr) = 
    h^{-1}\left(\frac{1}{n} \sum\limits_{i=1}^{n} g\left(X_{i}\right)\right)
\end{equation*}

Полученная оценка $\theta^{*}$~--- \textit{оценка метода моментов} для параметра $\theta$. 
Чаще всего берут $g(y)=y^{k}$. 
В этом случае, при условии обратимости функции~$h$ на~$\Omega$:
\begin{equation*}
    \ExpTh X_{1}^{k}=h(\theta), \quad 
    \theta = h^{-1}\left(\ExpTh X_{1}^{k}\right), \quad 
    \theta^{*} = h^{-1}\bigl( \overline{X^{k}} \bigr) = 
    h^{-1}\left(\frac{1}{n} \sum\limits_{i=1}^{n} X_{i}^{k}\right)
\end{equation*}

\begin{exmp}
    Рассмотрим равномерное распределение $\Uniform_{[0, \theta]}$. 
    Найдём оценку метода моментов для параметра $\theta$ по первому моменту:
    \begin{equation*}
        \ExpTh X_{1}=\frac{\theta}{2}, \quad 
        \theta = 2\,\ExpTh X_{1}, \quad 
        \theta_{1}^{*}=2 \SampleMean
    \end{equation*}
    Найдём оценку метода моментов k по $k$-му моменту:
    \begin{equation*}
        \ExpTh X_{1}^{k} =
        \int\limits_{0}^{\theta} y^{k} \frac{1}{\theta} \, dy = 
        \frac{\theta^{k}}{k+1}, \quad 
        \theta=\sqrt[k]{(k+1) \ExpTh X_{1}^{k}}, \quad 
        \theta_{k}^{*}=\sqrt[k]{(k+1) \overline{X^{k}}}
    \end{equation*}
\end{exmp}

\begin{thm*}
    Пусть $\theta^{*}=h^{-1}\bigl(\overline{g(\SampleX)}\bigr)$~--- оценка параметра $\theta$, полученная методом моментов, причём функция $h^{-1}$ непрерывна. 
    Тогда оценка $\theta^{*}$ состоятельна.
\end{thm*}

\begin{proof}
    По ЗБЧ Хинчина имеем:
    \begin{equation*}
        \overline{g(\SampleX)} = 
        \frac{1}{n} \sum\limits_{i=1}^{n} g\left(X_{i}\right) 
        \xrightarrow[n \to \infty]{\MyPr} \ExpTh g\left(X_{1}\right) = 
        h(\theta)
    \end{equation*}

    Ввиду непрерывности функции $h^{-1}$:

    \begin{equation*}
        \theta^{*} = h^{-1}\bigl(\overline{g(\SampleX)}\bigr) 
        \xrightarrow[n \to \infty]{\MyPr} h^{-1}\bigl(\ExpTh g\left(X_{1}\right)\bigr) = 
        h^{-1}\bigl(h(\theta)\bigr) = \theta.
    \end{equation*}
\end{proof}

\begin{defn}
    \textit{Асимптотически нормальная оценка} параметра $\theta$ с коэффициентом $\sigma^{2}(\theta)$~--- оценка $\theta^{*}$, т.ч. при $n \to \infty$ имеет место слабая сходимость к стандартному нормальному распределению: 
    \begin{equation*}
        \cfrac{\sqrt{n}(\theta^{*}-\theta)}{\sigma(\theta)} \xrightarrow[n \to \infty]{\text{d}} \Normal_{0, 1}.
    \end{equation*}
\end{defn}

\begin{lem}
    Пусть функция $g(y)$ такова, что $0 \neq \VarTh g\left(X_{1}\right)<\infty$. 
    Тогда статистика $\overline{g(\SampleX)}$ является асимптотически нормальной оценкой для $\ExpTh g\left(X_{1}\right)$ с коэффициентом $\sigma^{2}(\theta)=\VarTh g\left(X_{1}\right)$:

    \begin{equation*}
        \sqrt{n} \, \cfrac{\overline{g(\SampleX)}-\ExpTh g\left(X_{1}\right)}{\sqrt{\VarTh g\left(X_{1}\right)}} 
        \xrightarrow[n \to \infty]{\text{d}} \Normal_{0, 1}
    \end{equation*}
\end{lem}

\begin{proof}
    Следует непосредственно из ЦПТ.
\end{proof}

\begin{rmrk}
    Следующая теорема утверждает асимптотическую нормальность оценок вида

    \begin{equation*}
        \theta^{*} = 
        H \bigl( \overline{g(\SampleX)} \bigr) = 
        H\left(\cfrac{g\left(X_{1}\right)+\ldots+g\left(X_{n}\right)}{n}\right)
    \end{equation*}
    которые обычно получаются при использовании метода моментов, при этом всегда $\theta = H \bigl(\ExpTh g\left(X_{1}\right) \bigr)$.
\end{rmrk}

\begin{thm*}
    Пусть функция $g(y)$ такова, что $0 \neq \VarTh g\left(X_{1}\right)<\infty$, функция $H(y)$ дифференцируема в точке $a=\ExpTh g\left(X_{1}\right)$ и её производная в этой точке $H^{\prime}(a)=\left.H^{\prime}(y)\right|_{y=a}$ отлична от нуля. 
    Тогда оценка $\theta^{*}=H\bigl(\overline{g(\SampleX)}\bigr)$ является асимптотически нормальной оценкой для параметра $\theta = H\bigl(\ExpTh g\left(X_{1}\right)\bigr) = H(a)$ с коэффициентом асимптотической нормальности $\sigma^{2}(\theta)=\bigl(H^{\prime}(a)\bigr)^{2} \cdot \VarTh g\left(X_{1}\right)$.
\end{thm*}

\begin{proof}
    Согласно ЗБЧ последовательность $\overline{g(\SampleX)}$ стремится к $a=\ExpTh g\left(X_{1}\right)$ по вероятности с ростом $n$: Функция

    \begin{equation*}
        G(y)=\left\{\begin{array}{ll}
        \cfrac{H(y)-H(a)}{y-a}, & y \neq a \\
        H^{\prime}(a), & y=a
        \end{array}\right.  
    \end{equation*}
    по условию непрерывна в точке $a$: 
    поскольку сходимость по вероятности сохраняется под действием непрерывной функции, получим,
    что $G\bigl(\overline{g(\SampleX)}\bigr) \xrightarrow[n \to \infty]{\MyPr} G(a)=H^{\prime}(a)$.

    Заметим также, что по вышеприведённой лемме величина ${\sqrt{n}\bigl(\overline{g(\SampleX)} - a\bigr)}$ слабо сходится
    к нормальному распределению $\Normal_{0, \VarTh g(X_{1})}$: 
    пусть $\xi$~--- случайная величина из этого распределения. Тогда

    \begin{equation*}
        \sqrt{n} \Bigl(H\bigl(\overline{g(\SampleX)}\bigr) - H(a)\Bigr) = 
        \sqrt{n} \bigl(\overline{g(\SampleX)} - a\bigr) \cdot G\bigl(\overline{g(\SampleX)}\bigr) 
        \xrightarrow[n \to \infty]{\text{d}} \xi \cdot H^{\prime}(a)
    \end{equation*}

    Мы использовали следующее свойство слабой сходимости: если $\xi_{n} \xrightarrow[n \to \infty]{\text{d}} \xi$ и $\eta_{n} \xrightarrow[n \to \infty]{\MyPr} c=const$, то $\xi_{n} \eta_{n} \xrightarrow[n \to \infty]{\text{d}} c \xi$.
    Но распределение случайной величины $\xi \cdot H^{\prime}(a)$ есть $\Normal_{0,(H^{\prime}(a))^{2} \cdot \VarTh g(X_{1})}$, откуда следует

    \begin{equation*}
        \sigma^{2}(\theta)=\bigl(H^{\prime}(a)\bigr)^{2} \cdot \VarTh g\left(X_{1}\right).
    \end{equation*}
\end{proof}
  % Метод моментов. Свойства оценок, полученных методом моментов
\section{Метод максимального правдоподобия. Свойства оценок максимального правдоподобия}

\begin{defn}
    \textit{Оценка максимального правдоподобия $\theta^{*}(\SampleX)$ параметра $\theta$}~--- точка параметрического множества $\Theta$, в которой функция правдоподобия $L(\SampleX,\theta)$ при заданной реализации выборки $\boldsymbol{x}$ достигает максимума, т.е.:
    \begin{equation*}
        L(\boldsymbol{x}, \theta^{*})=\max\limits_{\theta \in \Theta} L(\boldsymbol{x}, \theta)
    \end{equation*}
\end{defn}

\begin{rmrk}
    Поскольку функция $\ln y$ монотонна, то точки максимума функций $L(\SampleX,\theta)$ и $\ln L(\SampleX,\theta)$ совпадают.
\end{rmrk}

Если для каждого $X$ максимум функции правдоподобия достигается во внутренней точке $\Theta$, и $L(\SampleX,\theta)$ дифференцируема по $\theta$, то оценка максимального правдоподобия $\theta^{*} = \theta^{*}(\SampleX)$ удовлетворяет уравнению:

\begin{equation*}
    \frac{\partial \ln L(\boldsymbol{x}, \theta)}{\partial \theta} = 0
\end{equation*}

Если $\theta$~--- векторный параметр: $\theta=\left(\theta_{1}, \ldots, \theta_{n}\right)$, то это уравнение заменяется системой уравнений:

\begin{equation*}
    \frac{\partial \ln L(\boldsymbol{x}, \theta)}{\partial \theta_{i}}=0,~ i=\overline{1, n} 
\end{equation*}

\begin{exmp}
    Пусть дана выборка $\SampleX= \Sample, \: X_i \sim \Binom_{1, \theta}$.
    Найдём оценку максимального правдоподобия для $\theta$.
    Для этого запишем функцию правдоподобия и исследуем её (точнее, её логарифм) на экстремум:
    \begin{gather*}
        L(\SampleX, \theta) = \prod\limits_{i = 1}^{n} \MyPrTh\bigl(X_i = x_i\bigr)= \theta^{\sum\limits_{i=1}^{n} X_i} (1 - \theta)^{n - \sum\limits_{i=1}^{n} X_i} \\
        \ln L(\SampleX, \theta) = \sum\limits_{i=1}^{n} X_i \ln \theta + \left(n - \sum\limits_{i=1}^{n} X_i\right) \ln(1 - \theta) \\
        \frac{\partial}{\partial \theta} \ln L(\SampleX, \theta) = \frac{\sum\limits_{i=1}^{n} X_i}{\theta} - \frac{n - \sum\limits_{i=1}^{n} X_i}{1 - \theta} = \\
        = \frac{(1 - \theta)\sum\limits_{i=1}^{n} X_i - \theta\left(n - \sum\limits_{i=1}^{n} X_i\right)}{\theta (1 - \theta)} = \\
        = \frac{\sum\limits_{i=1}^{n} X_i - n\theta}{\theta(1 - \theta)}
    \end{gather*}
    Приравнивая числитель к нулю, получаем, что $\theta^{*} = \SampleMean$~--- стационарная точка.
    При этом если $\theta > \SampleMean$, то производная отрицательна, а если $\theta < \SampleMean$~--- положительна, т.е. это локальный максимум.
    Значит, это в самом деле оценка максимального правдоподобия для параметра $\theta$.
\end{exmp}

\begin{thm*}
    Если существует эффективная оценка $\widetilde{T}(\SampleX)$ скалярного параметра $\theta$, то она совпадает с оценкой максимального правдоподобия.
\end{thm*}

\begin{proof}
    Если оценка $\widetilde{T}(\SampleX)$ скалярного параметра $\theta$ эффективна, то $\widetilde{T}(\SampleX)$ линейно выражается через функцию вклада (следствие неравенства Рао"--~Крамера):

    \begin{gather*}
        \widetilde{T}(\SampleX) - \theta = a_n(\theta) U(\SampleX, \theta) = a_n(\theta) \frac{\partial \ln L(\SampleX, \theta)}{\partial \theta}
    \end{gather*}
    Но если мы подставим вместо $\theta$ оценку максимального правдоподобия $\theta^{*}$, то правая часть обнулится.
    А значит, $\widetilde{T}(\SampleX) = \theta^{*}$.
\end{proof}

\begin{thm*}
    Если $T(\SampleX)$~--- достаточная статистика, а оценка максимального правдоподобия $\theta^{*}$ существует и единственна, то она является функцией от $T(\SampleX)$.
\end{thm*}

\begin{proof}
    Из критерия факторизации следует, что если $T=T(\SampleX)$ достаточная статистика, то имеет место представление:
    \begin{equation*}
        L(\SampleX, \theta)=g(T(\SampleX), \theta) h(\SampleX)
    \end{equation*}

    Таким образом, максимизация~$L(\SampleX,\theta)$ сводится к максимизации $g\bigl(T(\SampleX), \theta\bigr)$ по~$\theta$.
    Следовательно, $\theta^{*}$ есть функция от~$T(\SampleX)$.
\end{proof}

\begin{namedthm}[Инвариантность метода максимального правдоподобия]
    Пусть $f\colon \Theta \mapsto Y$~--- некоторая биективная функция.
    Тогда, если $\theta^*$ есть оценка максимального правдоподобия для $\theta$, то $f(\theta^*)$~--- оценка максимального правдоподобия для $f(\theta)$.
\end{namedthm}

\begin{proof}
    \begin{equation*}
        L(\boldsymbol{x}, \theta^*) = 
        \sup_{\theta \in \Theta} L(\boldsymbol{x}, \theta) =
        \sup_{y \in Y} L(\boldsymbol{x}, f^{-1}(y)) =
        L(\boldsymbol{x}, f^{-1}(y^*))
    \end{equation*}
    Тогда $\theta^* = f^{-1}(y^*)$ и $y^* = f(\theta^*)$.
\end{proof}

\begin{defn}
    \textit{Асимптотически эффективная оценка} параметра $\tau(\theta)$~--- оценка $\tau^{*}$:
    \begin{equation*}
        \VarTh \tau^{*} \cdot \frac{i_{n}(\theta)}{\bigl(\tau^{\prime}(\theta)\bigr)^{2}} 
        \xrightarrow[n \to \infty]{} 1,~ \text{где}~ i_{n}(\theta) = 
        \ExpTh\left(\frac{\partial \ln L(X, \theta)}{\partial \theta}\right)^{2} = 
        \ExpTh(U^{2}\bigl(X, \theta)\bigr)
    \end{equation*}
\end{defn}

\begin{thm*}
    Пусть выполнены следующие условия:
    \begin{enumerate}
        \item Функция правдоподобия $L(\SampleX, \theta)$ удовлетворяет условиям регулярности для первых двух производных;
        \item Существует единственная оценка максимального правдоподобия $\theta^{*}$ для всех $\theta$, которая достигается во внутренней точке $\Theta$.
    \end{enumerate}
    Тогда оценка $\theta^{*}$:
    \begin{enumerate}
        \item асимптотически несмещена;
        \item состоятельна;
        \item асимптотически эффективна;
        \item асимптотически нормальна.
    \end{enumerate}
\end{thm*}  % Метод максимального правдоподобия. Свойства оценок максимального правдоподобия
\section{Интервальное оценивание: Центральная статистика и использование точечной оценки}

\begin{defn}
    \textit{Доверительный интервал} для параметра $\theta$ с коэффициентом доверия (или надёжности) $\gamma \in (0, 1)$~--- интервал $\bigl(T_1(\SampleX), T_2(\SampleX)\bigr)$, где $T_i$~--- статистики, т.ч.:
    \begin{enumerate}
        \item $T_1(\SampleX) \overset{\text{п.н.}}{\leqslant} T_2({\SampleX})$;
        \item $\MyPrTh \Bigl(T_1(\SampleX) \leqslant \theta \leqslant T_2(\SampleX) \Bigr) \geqslant \gamma$.
    \end{enumerate}
\end{defn}

\begin{exmp}
    Пусть $\Sample$~--- выборка из $\Normal_{\theta, 1}$. Тогда
    \begin{equation*}
        \theta^{*}
        = \SampleMean
        = \frac{1}{n} \sum\limits_{i=1}^{n} X_{i} \sim \Normal_{\theta, \frac{1}{n}}
        \; \xrightarrow[n \to \infty]{\text{d}} \; (\SampleMean-\theta) \sqrt{n} \sim \Normal_{0;1}
    \end{equation*}

    Для величины, имеющей стандартное нормальное распределение, можно построить доверительный интервал следующим образом: находим такое $t_{\gamma / 2}$, что
    \begin{equation*}
        \MyPrTh\Bigl(\bigl|(\SampleMean-\theta) \sqrt{n}\bigr| \leqslant t_{\gamma / 2}\Bigr) = \gamma.
    \end{equation*}

    Решаем уравнение относительно $\theta$ и получаем
    \begin{equation*}
        \MyPrTh\left(\SampleMean-\cfrac{t_{\gamma / 2}}{\sqrt{n}} \leqslant \theta \leqslant \SampleMean+\cfrac{t_{\gamma / 2}}{\sqrt{n}}\right)=\gamma.
    \end{equation*}

    Таким образом, $ \left(\SampleMean-\cfrac{t_{\gamma / 2}}{\sqrt{n}}, \: \SampleMean+\cfrac{t_{\gamma / 2}}{\sqrt{n}} \right)$~--- $\gamma$-доверительный интервал для параметра $\theta$.
\end{exmp}

Длина построенного нами в примере интервала~--- $2\, \cfrac{t_{\gamma / 2}}{\sqrt{n}}$.
Она зависит лишь от размера выборки.
Однако в общем случае длина доверительного интервала является случайной величиной~--- $T_2(X) - T_1(X)$.
Вместе с тем очевидно, что нам хотелось бы сделать интервал как можно более коротким.
Рассмотрим достотаточно общий способ построения кратчайших доверительных интервалов.
Для этого нам понадобится ввести следующее определение.

\begin{defn}
    \textit{Центральная статистика}~--- функция $G(X,\theta)$, т.ч.:
    \begin{enumerate}
        \item $G(X,\theta)$ непрерывна и строго монотонна по $\theta$ при любой фиксированной реализации выборки $\boldsymbol{x}$. 
        \item $F_G (t) \equiv \MyPrTh \bigl(G(X, \theta) < t\bigr)$ непрерывна и не зависит от $\theta$.
    \end{enumerate}
\end{defn}

\begin{rmrk}
    Формально определённая выше величина не является статистикой, т.к. зависит от неизвестного параметра $\theta$.
\end{rmrk}

\subsubsection{Построение кратчайшего доверительного интервала с помощью центральной статистики:}

\begin{enumerate}
    \item Найдём $g_{1}, g_{2} \in \Real$, т.ч.
        \begin{equation*}
            \MyPrTh\Bigl(g_{1} \leqslant G(X, \theta) \leqslant g_{2}\Bigr)=\gamma~~\AllTh \quad \iff \quad  F_{G}(g_{2} + 0) - F_{G}(g_{1})=\gamma.
        \end{equation*}
    \item Пусть для определённости $G(X,\theta)$ возрастает по $\theta$. 
        Тогда существует обратная по $\theta$ функция $G^{-1}(X, g)$ и из условий
        \begin{equation*}
            %\left\{\begin{array}{l}
            %G(X, \theta) \leqslant \gamma_{2} \\
            %G(X, \theta) \geqslant \gamma_{1}
            %\end{array}\right.
            \begin{cases}
                G(X, \theta) \leqslant & g_{2} \\
                G(X, \theta) \geqslant & g_{1}
            \end{cases}
        \end{equation*}
        можно найти статистики
        \begin{equation*}
            %\left\{\begin{array}{l}
            %    T_{2}(\SampleX): G(X, T_{2}(\SampleX))=\gamma_{2} \\ 
            %    T_{1}(\SampleX): G(X, T_{1}(\SampleX))=\gamma_{1}
            %\end{array} 
            \begin{cases}
                T_{2}(\SampleX)\colon G\bigl(X, T_{2}(\SampleX)\bigr) = g_{2} \\ 
                T_{1}(\SampleX)\colon G\bigl(X, T_{1}(\SampleX)\bigr) = g_{1}
            \end{cases}
            \iff T_{1}(\SampleX) \leqslant \theta \leqslant T_{2}(\SampleX) 
        \end{equation*}
        откуда $\MyPrTh\Bigl(T_{1}(\SampleX) \leqslant \theta \leqslant T_{2}(\SampleX)\Bigr) \geqslant \gamma~ \AllTh$.
    \item 
        Минимизируем длину получившегося интервала.
        В общем случае это сводится к поиску условного экстремума
        \begin{equation*}
            \Bigl| G^{-1}(\SampleX, g_2) - G^{-1}(\SampleX, g_1) \Bigr| = 
            \Bigl| T_2(\SampleX) - T_1(\SampleX) \Bigr| \to 
            \min_{F_{G}(g_2 + 0) - F_{G}(g_1) \geqslant \gamma}
        \end{equation*}
\end{enumerate}

Поиск условного экстремума зачастую является сложной задачей.
Порой вместо минимизации длины интервала минимизируют среднюю длину $\ExpTh \bigl[ T_2(\SampleX) - T_1(\SampleX) \bigr]$, 
или даже отношение $\frac{g_2}{g_1}$ (считая, что $g_2 > g_1$).

\vspace{5mm}
Можно использовать и другой подход, который позволяет избежать решения сложных оптимизационных задач.
Согласно определению, параметр попадает в доверительный интервал с вероятностью $\gamma$.
Мы можем распределить оставшиеся $1 - \gamma$ поровну, чтобы параметр был левее или правее нашего интервала с равной вероятностью.

\begin{defn}
    \textit{Центральный доверительный интервал} для параметра $\theta$ с коэффициентом доверия $\gamma \in (0, 1)$~--- 
    интервал $\bigl(T_1(\SampleX), T_2(\SampleX)\bigr)$, т.ч.:
    \begin{gather*}
        \MyPrTh\Bigl(T_{1}(\SampleX) > \theta\Bigr)=\cfrac{1-\gamma}{2} \\
        \MyPrTh\Bigl(T_{1}(\SampleX) \leqslant \theta \leqslant T_{2}(\SampleX)\Bigr)=\gamma \\
        \MyPrTh\Bigl(T_{2}(\SampleX) < \theta\Bigr)=\cfrac{1-\gamma}{2}
    \end{gather*}
\end{defn}

%В этом случае при наличии центральной статистики построение интервала сводится просто к поиску квантилей распределения $F_G(x)$ соответствующих порядков: $\frac{1 - \gamma}{2}$ и $\frac{1 + \gamma}{2}$.
В этом случае при наличии центральной статистики мы однозначно находим $g_1, g_2$ как квантили распределения $F_G(x)\colon g_1 = F_G^{-1}\left(\frac{1 - \gamma}{2}\right), \, g_2 = F_G^{-1}\left(\frac{1 + \gamma}{2}\right)$, 
и остаётся лишь найти $T_1(\SampleX) = G^{-1}(\SampleX, g_1), \; T_2(\SampleX) = G^{-1}(\SampleX, g_2)$ (или наоборот, если $G(\SampleX, \theta)$ убывает по $\theta$).

\subsubsection{Построение доверительного интервала с помощью точечной оценки}
Порой бывает невозможно найти центральную статистику, так как функция распределения имеет слишком сложный вид~--- 
например, в случае бернуллиевского или пуассоновского распределения.
В таком случае можно использовать метод точечной оценки.

\begin{enumerate}
    \item Пусть $T(\SampleX)$~--- точечная оценка $\theta$. 
        Обозначим $F_{T}(t, \theta)={\MyPrTh\Bigl(T(\SampleX)<t\Bigr)}$. 
        Предположим, что $F_{T}(t,\theta)$~--- непрерывная и строго монотонная функция~$\theta$ при любом фиксированном~$t$. 
        Найдём такие $t_1(\theta), t_2(\theta)$, что ${\MyPrTh \Bigl( t_1(\theta) \leqslant T(\SampleX) \leqslant t_2(\theta) \Bigr) \geqslant \gamma}$.
        Для однозначности будем выбирать их так, чтобы
        \begin{equation*}
            \begin{cases}
                \MyPrTh \Bigl( T(\SampleX) > t_2(\theta) \Bigr) \leqslant \cfrac{1 - \gamma}{2} \\ 
                \MyPrTh \Bigl( T(\SampleX) < t_1(\theta) \Bigr) \leqslant \cfrac{1 - \gamma}{2} 
            \end{cases}
            \iff 
            \begin{cases}
                1 - F_{T}\bigl(t_2(\theta), \theta\bigr) \leqslant \cfrac{1 - \gamma}{2} \\
                F_{T}\bigl(t_1(\theta), \theta\bigr) \leqslant \cfrac{1 - \gamma}{2}
            \end{cases}
        \end{equation*}
        Т.е. речь идёт о построении центрального доверительного интервала.
        Подразумевается, что $t_1(\theta)$~--- наибольшее, а $t_2(\theta)$~--- наименьшее значения $T(\SampleX)$, удовлетворяющие этим требованиям.
        Если наблюдаемая случайная величина имеет непрерывное распределение, то неравенства заменяются на равенства.
        
    \item Рассмотрим вспомогательную лемму, верную в абсолютно непрерывном случае:
        \begin{lem}
            Если $F_{T}\bigl(t, \theta\bigr)$ возрастает по $\theta$, то $t_{1}(\theta)$ и $t_{2}(\theta)$ убывают. 
            Если же $F_{T}\bigl(t, \theta\bigr)$ убывает по $\theta$, то $t_{1}(\theta)$ и $t_{2}(\theta)$ возрастают.
        \end{lem}
        \begin{proof}
            Докажем для $t_1$, для $t_2$ аналогично.
            Пусть $F_{T}(t, \theta)$ возрастает.
            От противного: предположим, что $\exists \theta_{1}<\theta_{2} \colon t_{1}\left(\theta_{1}\right) \leqslant t_{1}\left(\theta_{2}\right)$.
            Используя то, что $F_{T}\bigl(t, \theta\bigr)$ возрастает по $\theta$ и, как всякая функция распределения, не убывает по $t$, получим:
            \begin{equation*}
                \frac{1-\gamma}{2} = 
                F_{T}\bigl(t_{1}(\theta_{1}), \theta_{1}\bigr) < 
                F_{T}\bigl(t_{1}(\theta_{1}), \theta_{2}\bigr) \leqslant 
                F_{T}\bigl(t_{1}(\theta_{2}), \theta_{2}\bigr) = 
                \frac{1-\gamma}{2}
            \end{equation*}
            Полученное противоречие завершает доказательство.
        \end{proof}
    \item %Из леммы следует, что для любого $\theta$ в абсолютно непрерывном случае
        Заключительный шаг:
        \begin{equation*}
            \begin{aligned}
                t_{1}(\theta) < T(\SampleX)
                \iff \theta > \varphi_{1} \bigl(T(\SampleX)\bigr)
                \implies 
                \MyPrTh\Bigl(\theta>\varphi_{1} \bigl(T(\SampleX)\bigr)\Bigr)
                = \cfrac{1-\gamma}{2} \\
                t_{2}(\theta) > T(\SampleX) 
                \iff \theta < \varphi_{2} \bigl(T(\SampleX)\bigr) 
                \implies 
                \MyPrTh\Bigl(\theta < \varphi_{2}\bigl(T(\SampleX)\bigr)\Bigr)
                = \cfrac{1-\gamma}{2} \\
                \implies 
                \MyPrTh \Bigl(\underbrace{\varphi_{2} \bigl(T(\SampleX)\bigr)}_{T_{1}(\SampleX)} 
                \leqslant \theta 
                \leqslant \underbrace{\varphi_{1}\bigl(T(\SampleX)\bigr)}_{T_{2}(\SampleX)} \Bigr)
                = \gamma
            \end{aligned}
        \end{equation*}
        где $\varphi_1, \varphi_2$~--- обратные к $t_1(\theta), t_2(\theta)$ функции. 
        Доказанная выше лемма является, по существу, обоснованием применимости метода в абсолютно непрерывном случае~--- 
        ведь если $t_1, t_2$ строго монотонны, то они обратимы.

        В дискретном случае придётся проверять обратимость вручную.
\end{enumerate}
  % Интервальное оценивание. Методы центральной статистики и использования точечной оценки
\section{Проверка гипотез. Лемма Неймана"--~Пирсона}

\begin{defn}
    \textit{Статистическая гипотеза $H$}~--- любое предположение о распределении наблюдаемой случайной величины.
    
    Гипотеза называется \textit{простой}, если в ней явно задаётся одно (не параметризованное) распределение. 
    Например, $H\colon X_i \sim \Normal_{0, 1}$. 
    В противном случае гипотеза называется \textit{сложной}.

    Как правило, рассматривается сразу две взаимоисключающие гипотезы.
    Одна из них называется \textit{основной} и обозначается $H_0$, а другая~--- \textit{альтернативной} и обозначается $H_1$.
\end{defn}

Гипотезы могут быть самыми разнообразными.
Приведём несколько примеров.

\begin{enumerate}
    \item 
        Гипотезы о виде распределения: \\
        Пусть производится $n$ независимых наблюдений над некоторой случайной величиной $\xi$ с неизвестной функцией распределения $F_{\xi}(x)$.
        Основная гипотеза~--- $H_0 \colon F_{\xi}(x) = F(x)$ или $H_0 \colon F_{\xi}(x) \in \SigAlg$, где $\SigAlg$~--- 
        некоторое подмножество в множестве всех распределений (как правило, $\SigAlg$ задаётся параметрически).
        %В этом случае гипотеза называется \textit{простой}, если она состоит из \textit{одного} значения параметра, и \textit{сложной} иначе;
    \item 
        Гипотезы о проверке однородности выборки: \\
        Произведено $k$ серий независимых наблюдений и получено $k$ выборок $\SampleX_1, \ldots, \SampleX_k$.
        Основная гипотеза состоит в том, что эти выборки извлечены из одного распределения, т.е.
        $H_0 \colon F_1(x) \equiv \ldots \equiv F_k(x)$, где $F_i$~--- функция распределения элементов $i$-й выборки.
    \item 
        Гипотеза независимости: \\
        Наблюдается двумерная случайная величина $\xi = (\xi_1, \xi_2)$ с неизвестной функцией распределения $F(x, y)$.
        Получена выборка $(X_1, Y_1), \ldots, (X_n, Y_n)$.
        Основная гипотеза заключается в том, что $\xi_1$ и $\xi_2$ независимы, 
        то есть $H_0\colon F(x, y) = F_{\xi_1}(x) F_{\xi_2}(y)$, где $F_{\xi_1}, F_{\xi_2}$~--- некоторые одномерные функции распределения.
        В общем случае можно рассматривать $k$-мерную случайную величину и проверять гипотезу независимости её компонент.
        
        %по выборке $(X_1, Y_1), \ldots, (X_n, Y_n)$ из $n$ независимых наблюдений пары случайных величин проверяется 
        %гипотеза $H_{0}=\{X_{i} \text { и } Y_{i} \text { независимы }\}$ при альтернативе $H_{1}=\{ \text { предположение неверно } \}$. 
        %Обе гипотезы являются сложными;
    \item 
        Гипотеза случайности: \\
        Результат эксперимента описывается $n$-мерной случайной величиной $X = \Sample$ с неизвестной функцией распределения $F_{X}(x_1, \ldots, x_n)$.
        Можно ли рассматривать $X$ как выборку из распределения некоторой случайной величины $\xi$ (т.е. являются ли компоненты $X_i$ независимыми и одинаково распределёнными)?
        В данном случае проверяется гипотеза случайности $H_0\colon F_{X}(x_1, \ldots, x_n) = F_{\xi}(x_1) \cdot \ldots \cdot F_{\xi}(x_n)$.

\end{enumerate}

\begin{rmrk}
    Может показаться, что в общем случае задача имеет такой вид:
    "<Дана выборка $\SampleX$. Верна ли гипотеза $H_0$?">
    Но корректнее говорить "<Согласуются ли наблюдаемые данные с высказанной гипотезой?">.

    В дальнейшем мы увидим, что процесс проверки гипотезы скорее напоминает доказательство неверности $H_0$.
    Мы оцениваем, насколько вероятна имеющаяся реализация выборки в предположении, что $H_0$ истинна.
    Если наблюдаемые значения достаточно маловероятны, то мы считаем, что гипотеза $H_0$ неправдоподобна и отвергаем её.
    В противном случае мы лишь можем сказать, что данные не противоречат высказанной гипотезе.
    
    Приведём модельный пример: пусть есть выборка из одного элемента $\SampleX= X_1$ и проверяется простая гипотеза $H_0\colon X_1 \sim U[-2, 2]$.
    Если наблюдаемая реализация выборки $x_1 = 3$, то наша гипотеза очевидно неверна, и мы можем спокойно её отвергнуть.
    Но если $x_1 = 1$, то мы не можем утверждать, что $H_0$ справедлива: 
    мы могли получить $x_1 = 1$ и из стандартного нормального распределения, и из пуассоновского, или из отрицательного биномиального.
\end{rmrk}

\medskip
В приведённом выше замечании мы уже сформулировали одну гипотезу, и, по сути, даже проверили её, 
интуитивно построив алгоритм принятия решения: если $x_1 \notin [-2, 2]$, то $H_0$ отвергается, иначе оснований считать её неверной нет.
Для того, чтобы проверять более сложные гипотезы, нам понадобится строить похожие алгоритмы, или критерии~--- правила, согласно которым \textit{по выборке} делается заключение о верности гипотезы.

\begin{defn}
    Критерий~--- это статистика $\varphi(\SampleX)$ (т.е. измеримая функция от выборки) со значениями из $[0, 1]$. 
    Трактуется как "<вероятность"> отвергнуть~$H_0$.
    
    Если $\varphi(\SampleX) \overset{\text{п.н.}}{\in} \{0, 1\}$, 
    то критерий называется \textit{нерандомизированным}, иначе~--- \textit{рандомизированным}.
\end{defn}

\begin{rmrk}
    С нерандомизированным критерием всё понятно: на каждой реализации выборки он даёт однозначный ответ, 
    принять ($\varphi(\boldsymbol{x}) = 0$) или отвергнуть ($\varphi(\boldsymbol{x}) = 1$) гипотезу.
    Но что делать, если $\varphi(\boldsymbol{x}) \in (0, 1)$?
    Тогда мы отвергаем гипотезу $H_0$ с вероятностью $\varphi(\boldsymbol{x})$.
    По сути, мы принимаем решение, подбрасывая (асимметричную) монетку.
    Понятно, что нерандомизированные критерии лучше, но, к сожалению, некоторые задачи не получается решить, отвечая лишь "<да"> и "<нет">.\\
\end{rmrk}

Каким образом строится критерий?

Выборка $\SampleX= \Sample$ объёма $n$~--- точка в пространстве $\Real^{n}$. 
Выделим такое множество $S \subset \Real^{n}$, что ${\MyPrTh \Bigl(\SampleX\in S \, | \, \{H_0 \text{ верна}\}\Bigr) \leqslant \alpha},$
где $\alpha \in (0, 1)$~--- некоторое наперёд заданное число.
Это множество называется \textit{критической областью} для гипотезы $H_0$. 
В этом случае критерий можно сформулировать следующим образом:
\begin{compactlist}
    \item $\SampleX\in S \implies$ отвергаем $H_0$;
    \item $\SampleX\notin S \implies$ нет оснований отвергать $H_0$, считаем её верной.
\end{compactlist}
Или $\varphi(\SampleX) = \Ind(\SampleX\in S)$ \textit{(нерандомизированный!)}.

По смыслу критическая область~--- это множество таких значений выборки, которые маловероятны при условии истинности $H_0$.
Поэтому при попадании выборки в критическую область основная гипотеза отвергается, как противоречащая статистическим данным.

\medskip
Конечно, если мы построили критерий и выбрали какую-то гипотезу, это не значит, что она стопроцентно верна.
Может оказаться, что мы отвергнули верную, или приняли ложную гипотезу.
Ошибки при проверке гипотез делятся на два типа.

\begin{defn}
    Говорят, что произошла \textit{ошибка 1-го рода (false positive)}, если критерий отверг верную гипотезу $H_0$. 
    Вероятность ошибки 1-го рода: 
    \begin{equation*}
        \alpha(S)=\MyPrTh\left(\SampleX\in S | \, H_{0}\right) = \MyPr{0}\left(\SampleX\in S\right)
    \end{equation*}
    Аналогично вероятность \textit{ошибки 2-го рода (false negative)}:
    \begin{equation*}
        \beta(S)=\MyPrTh\left(\SampleX\notin S | \, H_{1}\right)=\MyPr{1}\left(\SampleX\notin S\right)
    \end{equation*}
\end{defn}

\begin{rmrk}
    Вероятность ошибки 1-го рода также называется \textit{уровнем значимости критерия}.
    Выше мы говорили о том, что критическая область~--- это множество тех реализаций выборки, которые маловероятны при истинности основной гипотезы.
    При помощи $\alpha = \alpha(S)$ мы как раз и определяем, насколько маловероятно это событие.
    Естественным образом получается, что вероятность отвергнуть верную основную гипотезу равна $\alpha$.
\end{rmrk}

\begin{center}
    \begin{tabular}{|c|c|c|}
        \hline \multirow{2}{*} { Истинная гипотеза } & \multicolumn{2}{|c|} { Результат принятия решения } \\
        \cline {2-3} & $H_{0}$ принята & $H_{0}$ отклонена \\
        \hline $H_{0}$ & $1-\alpha$ & $\alpha$ \\
        \hline $H_{1}$ & $\beta$ & $1-\beta$ \\
        \hline
    \end{tabular}
\end{center}

Если $1 - \beta(S) < \alpha(S)$, то попасть в $S$ при условии истинности гипотезы $H_1$ труднее, чем при условии истинности гипотезы $H_0$, 
и тогда $S$~--- критическая область скорее для $H_1$. 
Хотелось бы, чтобы неравенство имело вид ${1 - \beta(S) \geqslant \alpha(S)}$.

\begin{defn}
    Критерий называется \textit{несмещённым}, если выполняется условие
    \begin{equation*}
        1 - \beta(S) \geqslant \alpha(S).
    \end{equation*}
\end{defn}

В общем случае не получается сделать сумму вероятностей ошибок обоих родов сколь угодно малой, 
так как задачи минимизации каждой из ошибок, как правило, противоречат друг другу.
Например, если наш критерий всегда будет отвергать основную гипотезу~--- $\varphi(\SampleX) \equiv 1$, то вероятность ошибки второго рода равна нулю.
И наоборот, всегда принимая основную гипотезу~--- $\varphi(\SampleX) \equiv 0$, мы никогда не совершим ошибку первого рода.

Конечно, можно привести пример, где есть идеальный критерий.
Пусть мы знаем, что наблюдаемая случайная величина распределена равномерно либо на отрезке $[0, 1]$ (основная гипотеза), либо на $[2, 3]$ (альтернатива).
Тогда критической областью будет $S = [2, 3]$, и критерий $\varphi(\SampleX) = \Ind(x_1 \in [2,3])$ будет иметь нулевые вероятности ошибок обоих родов.
Причиной тому то, что области значений исследуемых случайных величин не пересекаются. Однако такой пример весьма далёк от реальных ситуаций.

\medskip
Если мы не можем минимизировать и то и другое, логично зафиксировать один из параметров и оптимизировать оставшийся.
Обычно фиксируют уровень значимости, т.е. вероятность ошибки первого рода $\alpha$.

Рассмотрим так называемые \textit{параметрические гипотезы}.
Считаем, что выборка $\SampleX$ взята из некоторого параметрического семейства 
$\SigAlg = {\{F_{\theta}(x), \: \theta \in \Theta \subseteq \Real^r\}}, \; \theta = (\theta_1, \ldots, \theta_r)$. \\
\begin{tabular}{rl}
    Основная гипотеза~--- & $H_0\colon \theta \in \Theta_0, \: \Theta_0 \subset \Theta.$ \\
    Альтернатива~---   & $H_1\colon \theta \in \Theta_1, \: \Theta_1 = \Theta \setminus \Theta_0.$
\end{tabular}

\begin{defn}
    \textit{Мощность критерия}:
    \begin{equation*}
        W(\theta, \varphi) = \ExpTh \varphi(\SampleX) = \int\limits_{x \in \Real^n} \varphi(x) L(\mathbf{x}, \theta) \, dx.
        \footnote{Напоминаем, что при фиксированном $\theta$ функция правдоподобия является вероятностной мерой на выборочном пространстве.}
    \end{equation*}
\end{defn}

\begin{rmrk}
    Заметим, что если критерий $\varphi(\SampleX)$ нерандомизированный 
    (т.е. принимает только значения 0 и 1, однозначно решая, отвергнуть или принять основную гипотезу), 
    то
    \begin{equation*}
        W(\theta, \varphi) = \ExpTh \varphi(\SampleX) = 1 \cdot \MyPrTh (\SampleX\in S) + 0 \cdot \MyPrTh (\SampleX\notin S) = \MyPrTh(\SampleX\in S).
    \end{equation*} 
    Если $\theta \in \Theta_0$, то $\MyPrTh(\SampleX\in S | H_0) = \alpha(S)$. \\
    Если $\theta \in \Theta_1$, то $\MyPrTh(\SampleX\in S | H_1) = 1 - \MyPrTh(\SampleX\notin S | H_1) = 1 - \beta(S)$.
    Т.е. ошибки и первого, и второго рода можно выразить через функцию мощности.
\end{rmrk}

Разберёмся сначала с простыми гипотезами: $H_0\colon \theta = \theta_0, H_1\colon \theta = \theta_1$. 
Зададим $\alpha_0$ и будем иметь дело только с такими критериями, где $\alpha_{0} \geqslant \alpha(S)$ 
(т.е. вероятность ошибки первого рода не превосходит величины $\alpha_0$) и будем решать задачу $\beta(S) \to \min\limits_{S}$, 
что равносильно максимизации мощности при $\theta \in \Theta_1$ (в нашем случае $\Theta_1 = \{\theta_1\}$, и мы максимизируем $W(\theta_1, \varphi)$).

Получаем две эквивалентные задачи определения критической области $S$:
\begin{equation*}
    \begin{cases}
        \alpha(S) \leqslant \alpha_{0} \\
        \beta(S) \to \min\limits_{S}
    \end{cases}
    \iff~
    \begin{cases}
        W(\theta_0, \varphi) = \MyPr{\theta_0} (\SampleX\in S | H_0) \leqslant \alpha_0  \\
        W(\theta_1, \varphi) = \MyPr{\theta_1} (\SampleX\in S | H_1) \to \max\limits_{S} 
    \end{cases}
\end{equation*}

К сожалению, не всегда удаётся решить такую задачу, используя только нерандомизированные критерии $\varphi(\SampleX) = \Ind(\SampleX\in S)$.
Тяжело вздохнём и обратимся к рандомизированным.

Пусть дана выборка $\SampleX= \Sample$, относительно которой выдвинуто две простых параметрических гипотезы: 
$H_0\colon \theta = \theta_0$ и $H_1\colon \theta = \theta_1$.
Без ограничения общности будем предполагать, что существует плотность $f_{0}(x)$ для функции распределения $F_{0}(x) = F_{\theta_0}(x)$, соответствующей гипотезе $\theta = \theta_0$, 
и существует плотность $f_{1}(x)$ для функции распределения $F_{1}(x) = F_{\theta_1}(x)$.
В дискретном случае все результаты аналогичны.

Если верна гипотеза $H_1$, то функция правдоподобия выборки $X$ имеет вид:
\begin{equation*}
    L(\SampleX, \theta_1) =  L_{1}\left(\SampleX\right) = \prod_{i=1}^{n} f_{1}\left(X_{i}\right).
\end{equation*}

В противном случае 
\begin{equation*}
    L(\SampleX, \theta_0) = L_{0}\left(\SampleX\right) = \prod_{i=1}^{n} f_{0}\left(X_{i}\right).
\end{equation*}

\begin{rmrk}
    Вероятность ошибки первого и второго рода для рандомизированного критерия будем обозначать
    $\alpha(\varphi)$ и $\beta(\varphi)$ соответственно.
\end{rmrk}

Посчитаем вероятность ошибок, используя то, что при каждом $x \in \Real^n$ значение $\varphi(x)$~--- это вероятность отвергнуть $H_0$.
\begin{gather*}
    \MyPrTh\bigl(H_0 \text{ отвергнута} | H_0 \text{ верна}\bigr) = 
    \MyPr{0} \left(\overline{H}_{0}\right) = 
    \int\limits_{x \in \Real^{n}} \varphi(x) L_{0}(x)\, dx = \alpha(\varphi) \\
    \MyPrTh\bigl(H_0 \text{ принята} | H_0 \text{ неверна}\bigr) = 
    \MyPr{1} \left(H_{0}\right) = 
    \int\limits_{x \in \Real^{n}}(1-\varphi(x)) L_{1}(x)\, dx = \beta(\varphi) \\
\end{gather*}
Т.е. 
\begin{align*}
    W(\theta_0, \varphi) = &~\alpha(\varphi) \\
    W(\theta_1, \varphi) = &~1 - \beta(\varphi)
\end{align*}

Тогда задача построения статистического критерия %сводится к нахождению критической функции $\varphi(x)$ и 
будет формулироваться следующим образом:
\begin{equation*}
    \begin{cases}
        \alpha(\varphi) \leqslant \alpha_{0} \\
        \beta(\varphi) \to \min\limits_{\varphi}
    \end{cases}
    \iff
    \begin{cases}
        W(\theta_0, \varphi) = 
        \Exp_{\theta_0} \varphi(\SampleX) = 
        \Exp_{0} \varphi(\SampleX) \leqslant \alpha_{0} \\
        W(\theta_1, \varphi) = 
        \Exp_{\theta_1} \varphi(\SampleX) = 
        \Exp_{1} \varphi(\SampleX)\to \max\limits_{\varphi}
    \end{cases}
\end{equation*}
Таким образом, задача заключается в том, чтобы найти наиболее мощный критерий, 
когда вероятность ошибки первого рода не превосходит некоторого заданного порогового значения. 
Решение сформулированных задач даётся леммой Неймана"--~Пирсона.

\begin{namedthm}[Лемма Неймана"--~Пирсона]
    Пусть $\alpha_{0} \in(0, 1)$.
    Введём отношение функций правдоподобия $l(\SampleX) = \cfrac{L_{1}(\SampleX)}{L_{0}(\SampleX)}$.
    Тогда при фиксированной вероятности ошибки первого рода $\alpha_{0}$ наиболее мощный критерий (сокращенно НМК) имеет вид
    \begin{equation*}
        \varphi^{*}(x) = \begin{cases}
            1, & \text { если } l(\SampleX) > C \\
            \varepsilon, & \text { если } l(\SampleX) = C \\
            0, & \text { если } l(\SampleX) < C
        \end{cases}
    \end{equation*}
    где константы $C$ и $\varepsilon$ являются решениями уравнения "<вероятность ошибки первого рода для этого критерия равна $\alpha_0$">: 
    $\alpha\left(\varphi^{*}\right)=\alpha_{0}$.
\end{namedthm}

\begin{rmrk}
    Если для некоторой реализации выборки получится так, что $L_0(\boldsymbol{x}) = 0, L_1(\boldsymbol{x}) > 0$,
    то просто будем считать, что $l(\boldsymbol{x}) = +\infty > C$, и гипотезу $H_0$ надо отвергнуть.
\end{rmrk}

\begin{proof}
    \begin{enumerate}
        \item Покажем, что константы $C$ и $\varepsilon$ могут быть найдены из уравнения $\alpha\left(\varphi^{*}\right)=\alpha_{0}$. 
        Заметим, что
        
        \begin{equation*}
            \begin{aligned} \alpha(\varphi^{*})
            = P_{0}\Bigl(l(\SampleX) > C \Bigr)
            + \varepsilon P_{0} \Bigl(l(\SampleX) = C \Bigr) 
            \end{aligned}
        \end{equation*}

    %Если предположить, что $L_{0}(\SampleX)=0$, то

    %\begin{equation*}
    %    P_{0}\left\{L_{0}(\SampleX) = 0\right\} = \int\limits_{\left\{x: L_{0}(x)=0\right\}} L_{0}(x) \mu(d x)=0
    %\end{equation*}

    %и, следовательно, вышеприведённое равенство корректно. 
    %Поэтому рассмотрим случайную величину $\eta(\SampleX) = \cfrac{L_{1}(\SampleX)}{L_{0}(\SampleX)}$
    Отношение правдоподобий $l(\SampleX)$~--- случайная величина.
    Для удобства обозначим $\eta = l(\SampleX)$.
    Пусть $F_{\eta, H_0}(x)$~--- функция распределения этой величины в предположении, что $H_0$ верна.
 
    Тогда
    \begin{equation*}
        \alpha \left(\varphi^{*}\right) = 
        1-F_{\eta, H_0}(C) + \varepsilon\Bigl(F_{\eta, H_0}(C) - F_{\eta, H_0}(C - 0)\Bigr)
    \end{equation*}

    Пусть $g(C) = 1 - F_{\eta, H_0}(C)$, константу $C_{\alpha_{0}}$ можно выбрать так, чтобы было выполнено неравенство:
    \begin{equation*}
        g(C_{\alpha_{0}}) \leqslant \alpha_{0} \leqslant g(C_{\alpha_{0}} - 0)
    \end{equation*}
    Тогда
    \begin{equation*}
        \varepsilon_{\alpha_{0}} = 
        \begin{cases}
            0, & \text{ если }  g(C_{\alpha_{0}}) = g(C_{\alpha_{0}} - 0) \\
            \cfrac{\alpha_{0} - g\left(C_{\alpha_{0}}\right)}{g(C_{\alpha_{0}}-0)-g(C_{\alpha_{0}})} \in [0, 1], & \text{ если } g(C_{\alpha_{0}}) < g(C_{\alpha_{0}}-0)
        \end{cases}
    \end{equation*}

    В обоих случаях выполнено равенство:
    \begin{equation*}
        \alpha_{0} = 
        g(C_{\alpha_{0}}) + \varepsilon_{\alpha_{0}}\Bigl(g(C_{\alpha_{0}} - 0) - g(C_{\alpha_{0}})\Bigr) =
        \alpha\left(\varphi^{*}\right)
    \end{equation*}

    \item Докажем, что $\varphi^{*}(x)$~--- наиболее мощный критерий.

    Выберем любой другой критерий $\tilde{\varphi}(x)$ такой, что $\alpha(\tilde{\varphi}) \leqslant \alpha_{0}$, 
    и сравним ее с критерием $\varphi^{*}(x)$. 
    Заметим, что для любого $x$ справедливо неравенство:
    \begin{equation*}
        \bigl(\varphi^{*}(x)-\tilde{\varphi}(x)\bigr) \Bigl(L_{1}(x)-c_{\alpha_{0}} L_{0}(x)\Bigr) \geqslant 0
    \end{equation*}

    Тогда
    \begin{equation*}
        \int\limits_{\Real^{n}} \bigl(\varphi^{*}(x)-\tilde{\varphi}(x)\bigr) \Bigl(L_{1}(x)-c_{\alpha_{0}} L_{0}(x)\Bigr) \, dx \geqslant 0
    \end{equation*}

    Раскроем скобки и преобразуем:
    \begin{multline*}
        \int\limits_{\Real^{n}} \varphi^{*}(x) L_{1}(x) \, dx - \int\limits_{\Real^{n}} \tilde{\varphi}(x) L_{1}(x) \,dx \geqslant \\
        \geqslant C_{\alpha_{0}}\left(\int\limits_{\Real^{n}} \varphi^{*}(x) L_{0}(x) \,dx - \int\limits_{\Real^{n}} \tilde{\varphi}(x) L_{0}(x) \,dx\right) \geqslant 0
    \end{multline*}
    Последнее верно, т.к. $\alpha(\tilde{\varphi}) \leqslant \alpha_0 = \alpha(\varphi^*)$.
    
    \smallskip
    Следовательно, $W \left(\theta_1, \varphi^{*}\right) - W(\theta_1, \tilde{\varphi}) \geqslant C_{\alpha_{0}}\bigl(\alpha\left(\varphi^{*}\right)-\alpha(\tilde{\varphi})\bigr)$, откуда получаем неравенство:

    \begin{equation*}
        W\left(\theta_1, \varphi^{*}\right) \geqslant W(\theta_1, \tilde{\varphi})
    \end{equation*}
    Что и требовалось доказать.
    \end{enumerate}
\end{proof}

\begin{rmrk}
    Все вышесказанное относится к случаю, когда мы проверяем две простые гипотезы: основную $H_0\colon \theta = \theta_0$ и альтернативу $H_1\colon \theta = \theta_1$.
    В случае сложных гипотез следует зафиксировать некоторые $\theta_0 \in \Theta_0, \: \theta_1 \in \Theta_1$ и применить лемму Неймана"--~Пирсона уже для простых гипотез.
    Если в результате построенный критерий не зависит от значения $\theta_1$, то он называется \textit{равномерно наиболее мощным критерием} (сокращённо РНМК).
\end{rmrk}
 % Проверка гипотез. Лемма Неймана—Пирсона
\section{Критерии согласия Колмогорова и \texorpdfstring{$\chi^{2}$}{хи-квадрат}}

Пусть для наблюдаемого распределения $\MyPr{\xi}$ дана выборка $X_1, \ldots, X_n$, 
проверяется \textit{гипотеза о виде распределения} $H_{0}\colon F_{\xi}=F_{0}$, где $F_{0}$ известна; 
альтернатива $H_{1}: F_{\xi} \neq F_{0}$.

\subsubsection{Критерий согласия $\chi^{2}$ Пирсона}
Разобьём числовую ось на $k$ промежутков ${-\infty=a_{0}<a_{1}<\ldots<a_{k}=\infty}$, ${\Delta_{i}=\left(a_{i-1}, a_{i}\right]}$ и построим статистику $\overline{\chi}^{2}$:
\begin{equation*}
    \overline{\chi}^{2}(\SampleX) = 
    \sum\limits_{i=1}^{k} \frac{\left(\text{наблюдаемое} - \text{ожидаемое}\right)^{2}}{\text{ожидаемое}} = 
    \sum\limits_{i=1}^{k} \frac{\left(n_{i}-n p_{i}^{(0)}\right)^{2}}{n p_{i}^{(0)}},
\end{equation*}
где $n_i$~--- число зафиксированных наблюдений в $i$-м интервале,
${p_{i}^{(0)}=F_{0}\left(a_{i}\right)-F_{0}\left(a_{i-1}\right)}$~--- вероятность попадания наблюдения в $i$-й~интервал при выполнении гипотезы $H_0$, 
$n p_{i}^{(0)}$, соответственно, ожидаемое число попаданий в $i$-й интервал.

\bigskip
\noindent \textbf{Формулировка критерия для простой гипотезы:}

Рассмотрим простую гипотезу $H_0\colon F_{\xi}(x) = F_0(x)$.
\begin{compactlist}
    \item Если верна гипотеза $H_0$, то $\overline{\chi}^{2}\left(\SampleX\right) \xrightarrow[n \to \infty]{\text{d}} \chi^{2}_{k-1}$, 
    %где $\zeta$ подчиняется распределению $\chi^{2}$ с $k-1$ степенями свободы ($k$~--- число интервалов разбиения);
    где $k$~--- число интервалов разбиения.
    \item Если верна гипотеза $H_1$, то $\overline{\chi}^{2} \xrightarrow[n \to \infty]{\text{п.н.}} \infty$.
\end{compactlist}

Выберем уровень значимости (т.е. вероятность ошибки первого рода) $\alpha \in (0, 1)$. 
Область $(\chi^{2}_{k-1, 1 - \alpha}, \infty)$, где $\chi^{2}_{k-1, 1 - \alpha}$~--- квантиль порядка $1-\alpha$ распределения $\chi^{2}$ с $k-1$ степенями свободы, является критической для гипотезы $H_0$.

Правило проверки гипотез:
\begin{compactlist}
    \item Если $\overline{\chi}^{2} \left(\SampleX\right)>\chi^{2}_{k-1, 1 - \alpha}$, то $H_0$ отклоняется;
    \item Если $\overline{\chi}^{2} \left(\SampleX\right) \leqslant \chi^{2}_{k-1, 1 - \alpha}$, то для отклонения $H_0$ нет оснований.
\end{compactlist}
\medskip
%sec2-11
\begin{center}
    \begin{tikzpicture}
        \begin{axis}[
            height=8cm, width=15cm, 
            xmin=-0.5, ymin=-0.008,
            xmax=15, ymax=0.1, 
            axis line style = thick,
            axis lines = middle,
            enlargelimits=false, axis on top, ticks=none,
            ]
            \addplot[name path=bell, very thick, blue, domain=0:15,samples=100]{exp(-x/2)*x/8};
            \path [name path=flooor]
            (\pgfkeysvalueof{/pgfplots/xmin},0) -- (\pgfkeysvalueof{/pgfplots/xmax},0);
            \addplot [blue!20] fill between [of=bell and flooor, soft clip={domain=8:15}];
            \addplot[only marks, color=blue, thick, mark=*] plot coordinates {(8, 0)} node[black, below]{$c_{1 - \alpha}$};
            \node[draw=blue, text=blue, fill=white] at (9.4, 0.005) {\small$\alpha$};
        \end{axis}
    \end{tikzpicture}
\end{center}

Фактически критерий $\chi^{2}$ проверяет значимость расхождения эмпирических (наблюдаемых) и теоретических (ожидаемых) частот. 
Рассмотрим его применение на следующем примере.

\begin{exmp}
    При $4040$ бросках монеты математик Бюффон получил $n_1 = 2048$ выпадения "<герба"> и $n_2 = n - n_1 = 1992$ выпадения "<решки">.
    Проверим, согласуются ли данные с гипотезой о том, что монета симметрична, т.е. $H_0 \colon p = 1/2$.
    Мы исследуем бернуллиевскую случайную величину $\xi \sim \Bernoulli_{p} = \Binom_{1, p}$.
    Она принимает лишь значения $0$ и $1$.
    Выберем промежутки $(-0.5, 0.5)$ и $(0.5, 1.5)$ (или любые другие два, содержащие $0$ и $1$ соответственно).
    Согласно предположению, $p = p_1^{(0)} = p_2^{(0)} = 1/2$.
    Подсчитаем значение статистики $\chi^{2}_{\SampleX}$:
    \begin{equation*}
        \sum\limits_{i = 1}^{k} \frac{(n_i - np_i^{(0)})^2}{np_i^{(0)}} = 
        \frac{(n_1 - np)^2}{np} + \frac{(n_2 - np)^2}{np} = 
        \frac{28^2}{2020} + \frac{28^2}{2020} = \frac{784}{1010} \approx 0.776
    \end{equation*}
    Положим уровень значимости $\alpha = 0.05$ и найдём квантиль $\chi^2_{k-1, 1 - \alpha} = \chi^2_{1, 0.95} \approx 3.8415$.
    Сравниваем полученное: $0.776 < 3.8415$.
    Делаем вывод, что данные не противоречат гипотезе.
\end{exmp}

\noindent \textbf{Критерий $\chi^2$ для сложной гипотезы.}

Рассмотрим сложную гипотезу $H_0\colon F_{\xi}(x) \in \SigAlg_{\theta} = \bigl\{ F(x, \theta)\colon \theta \in \Theta\bigr\}$.
Как и в простом случае, сгруппируем данные в $k$ интервалов. 
Теоретические вероятности попадания в интервалы теперь не будут заданы однозначно, а представляют собой некоторые функции от параметра $\theta$.
Поэтому статистика имеет вид 
\begin{equation*}
    \overline{\chi^2} = \sum\limits_{i = 1}^{n} \frac{\left(n_i - n p_{i}^{(0)}(\theta)\right)^2}{n p_{i}^{(0)}(\theta)}
\end{equation*}
Эта статистика зависит от неизвестного параметра; следовательно, непосредственно использовать её для построения критерия пока нельзя~---
требуется сначала исключить неопределённость, связанную с неизвестным параметром $\theta$.
Для этого заменяют $\theta$ некоторый оценкой $\tilde{\theta} = \tilde{\theta}(\SampleX)$ и получают статистику
\begin{equation*}
    \overline{\chi^2} = \sum\limits_{i = 1}^{n} \frac{\left(n_i - n p_{i}^{(0)}(\tilde{\theta}) \right)^2}{n p_{i}^{(0)}(\tilde{\theta})}
\end{equation*}
Но, вообще говоря, теперь $p_{i}^{(0)}(\tilde{\theta})$ являются случайными величинами, и мы не можем утверждать, что распределение статистики $\overline{\chi^2}$ будет стремиться к $\chi^{2}_{k-1}$.
Более того, следует ожидать, что распределение этой статистики (если оно существует) будет зависеть от способа построения оценки $\tilde{\theta}$.

К счастью для нас, английский статистик Рональд Фишер ещё в 1924 году показал, что существуют методы оценивания параметра $\theta$, 
при котором предельное распределение имеет простой вид, а именно является распределением $\chi^{2}_{k - 1 - r}$, где $r$~--- размерность оцениваемого параметра.
Один из таких методов использует мультиномиальную оценку максимального правдоподобия.


\begin{exmp}
    Следующая задача возникла в связи с бомбардировками Лондона во время Второй мировой войны. 
    Для улучшения организации оборонительных мероприятий, необходимо было понять цель противника. 
    Для этого территорию города условно разделили сеткой из 24 горизонтальных и 24 вертикальных линий на 576 равных участков. 
    В течении некоторого времени в центре организации обороны города собиралась информация о количестве попаданий снарядов в каждый из участков. 
    В итоге были получены следующие данные:
    \begin{center}
        \begin{tabular}{|c|c|c|c|c|c|c|c|c|}
        \hline Число попаданий & 0 & 1 & 2 & 3 & 4 & 5 & 6 & 7 \\
        \hline Количество участков & 229 & 211 & 93 & 35 & 7 & 0 & 0 & 1 \\
        \hline
        \end{tabular}
    \end{center}
    
    %Гипотеза $H_0$: стрельба случайна (нет "<целевых"> участков).
    Сформулируем основную гипотезу: стрельба случайна (нет "<целевых"> участков).
    В таком случае количество попаданий в участок можно описать распределением Пуассона~--- 
    оно моделирует число событий, произошедших за фиксированное время, при условии, что данные события происходят с некоторой фиксированной средней интенсивностью и независимо друг от друга.
    Т.е. $H_0\colon F(x) \sim \Pois_{\lambda}$.\footnote{Пример взят из \textit{Лагутин М.Б.} Наглядная математическая статистика: учебное пособие.~--- 2-е изд., испр.~--- М.: БИНОМ. Лаборатория знаний, 2009.~--- стр. 278.}
    
    Высчитаем теоретические вероятности:
    \begin{equation*}
        p_i^{(0)} = \MyPr\{S=i\}=\frac{\lambda^{i}}{i !} e^{-\lambda},
    \end{equation*}
    где $S$~--- число попаданий, $\lambda \approx 0{,}924$ \textit{(мультиномиальная ОМП)}.
    
    Обозначим за $n_i$ количество участков, на которые пришлось $i$ попаданий, и составим новую таблицу для применения критерия.
    
    \begin{center}
        \begin{tabular}{|c|c|c|c|c|c|c|c|c|}
            \hline $i$ & 0 & 1 & 2 & 3 & 4 & 5 & 6 & 7 \\
            \hline $n_i$ & 229 & 211 & 93 & 35 & 7 & 0 & 0 & 1 \\
            $n_{i} \cdot p_{i}^{(0)}$ & 226{,}7 & 211{,}4 & 98{,}5 & 30{,}6 & 7{,}14 & 1{,}33 & 0{,}21 & 0{,}03 \\
            \cline { 6 - 9 }$n_{i} \cdot \tilde{p}_{i}^{(0)}$ & 228{,}6 & 211{,}3 & 97{,}6 & 30{,}1 & \multicolumn{4}{|c|} {8{,}46} \\
            \hline
        \end{tabular}
    \end{center}
    
    Прежде чем вычислять статистику $\overline{\chi}^{2}$, мы объединили 4 последних события с низкими частотами в одно (соответственно, $k=5$) 
    и пересчитали новые теоретические вероятности~$\tilde{p}_i^{(0)}$ и, соответственно, новые ожидаемые значения. 
    В этом случае $\overline{\chi}^{2} \approx 1{,}05$. 
    Т.к. $k=5$, то по таблице распределения $\chi^{2}$ находим соответствующий уровень значимости $\alpha = 0{,}79$. 
    Гипотеза о низкой точности стрельбы не отклоняется.
    Посмотрим теперь на квантили распределения $\chi^{2}_{k-1-r} = \chi^{2}_{3}$:
    \begin{center}
        \begin{tabular}{|c|c|c|c|c|c|c|c|c|c|}
            \hline $1 - \alpha$         & 0{,}025 & 0{,}05  & 0{,}1   & 0{,}2   & 0{,3}   & 0{,}5  & 0{,}9  & 0{,}95  & 0{,}99 \\
            \hline Квантиль $\chi^{2}_{3}$ & 0{,}216 & 0{,}352 & 0{,}584 & 1{,}005 & 1{,}424 & 2{,}366 & 6{,}2514 & 7{,}815 & 11{,}345 \\
            \hline
        \end{tabular}
    \end{center}
    Даже если мы выберем $\alpha = 0{,}7$ и будем отвергать верную основную гипотезу с вероятностью $0{,}7$, критерий $\chi^2$ всё равно примёт $H_0$.
    То есть, соответствие гипотезы с наблюдаемыми данными очень хорошее.
    
    Обратим внимание на необходимость объединения маловероятных промежутков: если оставить $k = 8$, то $\overline{\chi}^{2} \approx 32{,}6$, 
    что значительно велико даже на уровне $\alpha = 10^{-5}$. 
    Подобная ошибка критерия $\chi^{2}$ вероятна на всех выборках с низкочастотными событиями. 
    Проблема решается либо отбрасыванием, либо объединением данных событий. %(\textit{коррекция Йетса}).
\end{exmp}
Рекомендуемые условия применения критерия согласия $\chi^2$ Пирсона~--- $n \geqslant 50, n_i \geqslant 5 \; \forall i = \overline{1, k}$.
    
    
\subsubsection{Критерий Колмогорова}
Наложим дополнительное условие на исходную задачу проверки гипотезы $H_0\colon F_{\xi}(x) = F_0(x)$~--- $F_{0}(x) \in C(\Real)$.

Рассмотрим статистику Колмогорова:
\begin{equation*}
    D_{n}\left(\SampleX\right)=\sup\limits_{x \in R}\left|F_{n}^{*}(x)-F_{0}(x)\right|
\end{equation*}

Формулировка критерия:
\begin{compactlist}
    \item Если верна гипотеза $H_0$, то $D_{n}\left(\SampleX\right) \xrightarrow[n \to \infty]{\text{п.н.}} 0$;
    \item Если верна гипотеза $H_1$, т.е. $F_{\xi} \equiv G \neq F_{0}$, то
    \begin{equation*}
        % Дробь из стрелки и текста~--- это шедевр. Не могу удалить.
        % D_{n}\left(\SampleX\right) \frac{\text { п.н. }}{n \to \infty} \sup\limits_{x \in R}\left|G(x)-F_{0}(x)\right|>0
        D_{n}\left(\SampleX\right) \xrightarrow[n \to \infty]{\text{п.н.}} \sup\limits_{x \in R}\left|G(x)-F_{0}(x)\right|>0
    \end{equation*}
\end{compactlist}

\begin{lem}
Если гипотеза $H_0$ верна, и $F_{0}(x) \in C(\Real)$, то распределение статистики $D_{n}=\sup\limits_{x \in R}|F_{n}^{*}(x)-F_{0}(x)|$ не зависит от наблюдаемого распределения.
\end{lem}

При больших $n$ применяется асимптотический подход.
\begin{namedthm}[Теорема Колмогорова]
Если гипотеза $H_0$ верна, и $F_{0}(x) \in C(\Real)$, то имеет место сходимость:
\begin{equation*}
    P\left\{\sqrt{n} D_{n}\left(\SampleX\right) \leqslant z\right\} \xrightarrow[n \to \infty]{} K(z)=1+2 \sum\limits_{m=1}^{\infty}(-1)^{m} e^{-2 m^{2} z^{2}}
\end{equation*}
\end{namedthm}

Находим константу $d_{1-\alpha}$ как решение уравнения $K\left(d_{1-\alpha}\right)=1-\alpha$.

Правило проверки гипотез:
\begin{compactlist}
    \item Если $\sqrt{n} D_{n}\left(\SampleX\right) \in\left(d_{1-\alpha}, \infty\right)$, то гипотеза $H_0$ отвергается;
    \item Если $\sqrt{n} D_{n}\left(\SampleX\right) \notin\left(d_{1-\alpha}, \infty\right)$, то гипотеза $H_0$ принимается.
\end{compactlist}

\begin{exmp}
Среди 100 студентов ВМК была проведена контрольная по ТВиМС.
\begin{center}
    \begin{tabular}{|l|l|l|l|l|l|}
    \hline Количество решённых задач & 1  & 2  & 3  & 4  & 5 \\
    \hline Частота                   & 18 & 16 & 26 & 22 & 18 \\
    \hline
\end{tabular}
\end{center}
На уровне значимости $\alpha=0{,}2$ с помощью критерия Колмогорова определить, 
подчиняются ли данные выборки на интервале $[0, 5]$ равномерному закону распределения случайной величины.

Запишем теоретическую функцию распределения:
\begin{equation*}
    F_{0}(x) = \begin{cases}
        0, & x<0 \\
        x/5, & 0 \leqslant x \leqslant 5 \\
        1, & x>5    
    \end{cases}
\end{equation*}

Составим следующую таблицу:
\begin{center}
    \begin{tabular}{|c|c|c|c|c|}
        \hline $x_{i}$ & $F(x_{i})$ & $n_{i}$ & $F^{*}_{n}(x_{i})$ & $|F_{0}(x_{i})-F^{*}_{n}(x_{i})|$ \\
        \hline 1 & 0{,}2 & 18 & 0{,}18 & 0{,}02 \\
        \hline 2 & 0{,}4 & 16 & 0{,}34 & 0{,}06 \\
        \hline 3 & 0{,}6 & 26 & 0{,}6  & 0    \\
        \hline 4 & 0{,}8 & 22 & 0{,}82 & 0{,}02 \\
        \hline 5 &   1 & 18 &    1 & 0    \\
        \hline
    \end{tabular}
\end{center}
Отсюда $D_{n}(x)=\sup\limits_{x \in R}\left|F_{n}^{*})-F_{0}(x)\right| = 0{,}06$, $\sqrt{n}D_{n}(x) = 0{,}6$, 
что меньше критического значения $0{,}65$ функции Колмогорова при уровне значимости $\alpha=0{,}2$, следовательно, гипотеза о равномерном распределении принимается.
\end{exmp}
 % Критерии согласия Колмогорова и хи-квадрат
\section{Статистические выводы о параметрах нормального распределения. Распределения \texorpdfstring{$\chi^{2}$}{хи-квадрат} и Стьюдента. Теорема Фишера}

\begin{defn}
    Говорят, что случайная величина $\xi$ имеет \textit{гамма-распределение} с параметрами $\lambda > 0,~ \alpha > 0$ (или $\xi \sim \GammaDist_{\lambda, \alpha}$), 
    если $\xi$ имеет следующую плотность распределения:
    \begin{equation*}
        p_{\xi}(x) = \begin{cases}
            0, & x \leqslant 0; \\
            c \cdot x^{\alpha-1} e^{-\lambda x}, & x > 0,
        \end{cases}
    \end{equation*}
    где постоянная $c$ вычисляется из свойства нормировки плотности:
    \begin{equation*}
        1 = 
        \int\limits_{-\infty}^{\infty} p_{\xi}(x) d x = 
        c \int\limits_{0}^{\infty} x^{\alpha-1} e^{-\lambda x} d x = 
        \frac{c}{\lambda^{\alpha}} \int\limits_{0}^{\infty}(\lambda x)^{\alpha-1} e^{-\lambda x} d(\lambda x) = 
        \frac{c}{\lambda^{\alpha}} \Gamma(\alpha),
    \end{equation*}
    откуда $c=\lambda^{\alpha} / \Gamma(\alpha)$.
\end{defn}

\begin{lem}
    Пусть $\xi_{1}, \ldots, \xi_{n}$ независимы, и $\xi_i \sim \GammaDist_{\lambda, \alpha_i}, i=\overline{1,n}$. 
    Тогда их сумма $S_{n}=\xi_{1}+\ldots+\xi_{n} \sim \GammaDist_{\lambda, \alpha_1 + \ldots + \alpha_n}$.
\end{lem}
\begin{proof}
    Найдём для начала характеристическую функцию гамма-распределения:
    \begin{multline*}
        \int\limits_{-\infty}^{\infty} p_{\xi}(x) e^{itx} \, dx = 
        \int\limits_{-\infty}^{\infty} \frac{\lambda^\alpha \, x^{\alpha - 1}}{\Gamma(\alpha)} \, e^{-\lambda x}\, e^{itx} \,dx = 
        \left| \begin{array}{c}
            y  = -(\lambda - it)x \\
            dy = -(\lambda - it)dx
        \end{array} \right| = \\
        \int\limits_{-\infty}^{\infty} \frac{\lambda^\alpha \, y^{\alpha - 1}}{(\lambda - it)^{\alpha - 1} \, \Gamma(\alpha)} \, e^{-y} \frac{1}{\lambda - it} \,dy = 
        \frac{\lambda^\alpha}{(\lambda - it)^{\alpha}} \int\limits_{-\infty}^{\infty} \frac{y^{\alpha - 1}}{\Gamma(\alpha)} \, e^{-y} \, dy = 
        \left(\frac{1}{1 - \frac{it}{\lambda}} \right)^\alpha \!.
    \end{multline*}
    Теперь вспомним, что если $\xi$ и $\eta$ независимы, то $f_{\xi + \eta}(t) = f_{\xi}(t) f_{\eta}(t)$.
    Тогда, если у нас есть независимые случайные величины $\xi_1, \ldots, \xi_n, \: \xi_i \sim \Gamma(\lambda, \alpha_i)$, 
    то характеристическая функция суммы выражаются следующим образом:
    \begin{equation*}
        \left(\frac{1}{1 - \frac{it}{\lambda}} \right)^\alpha_1 \cdot \ldots \cdot \left(\frac{1}{1 - \frac{it}{\lambda}} \right)^\alpha_n = 
        \left(\frac{1}{1 - \frac{it}{\lambda}} \right)^{\alpha_1 + \ldots + \alpha_n}
    \end{equation*}

    По теореме Леви о непрерывности заключаем, что сумма имеет распределение $\Gamma(\lambda, \alpha_1 + \ldots + \alpha_n)$.
\end{proof}

\begin{lem}
    Если $\xi \sim \Normal_{0,1}$, то $\xi^2 \sim \GammaDist_{1/2, 1/2}$.
\end{lem}

\begin{proof}
    Найдём функцию распределения случайной величины~${\eta = \xi^2}$:
    \begin{gather*}
        \MyPr(\xi^2 < x) = \MyPr(|\xi| < \sqrt{x}) = 2 \, \MyPr(0 < x < \sqrt{x}) = \\
        2 \int\limits_{0}^{\sqrt{x}} \frac{1}{\sqrt{2 \pi}}\,  e^{u^2 /2} \,du = 
        \left| \begin{array}{cc}
            z = u^2    & u = \sqrt{z} \\
            dz = 2u du & du = \frac{dz}{2 \sqrt{z}} \\
            u = 0 \implies z = 0 & u = \sqrt{x} \implies z = x
        \end{array} \right| = \\
        = 2 \int\limits_{0}^{x} \frac{1}{\sqrt{2 \pi}} \, e^{-\frac{1}{2}z} \frac{dz}{2 \sqrt{z}} = 
        \int\limits_{0}^{x} \frac{\left(\frac{1}{2}\right)^{\frac{1}{2}} z^{\frac{1}{2} - 1}}{\sqrt{\pi}} e^{-z/2} dz.
    \end{gather*}
    Но это в точности функция гамма-распределения с параметрами $\lambda = \frac{1}{2}, \alpha = \frac{1}{2}$ (ведь $\Gamma\left(\frac{1}{2}\right) = \sqrt{\pi}$).
\end{proof}

\begin{crlr}
    Если $\xi_{1}, \ldots, \xi_{k}$ независимы и $\xi_i \sim \Normal_{0,1}$, то случайная величина $\chi^{2}=\xi_{1}^{2}+\ldots+\xi_{k}^{2} \sim \GammaDist_{1/2, k/2}$
\end{crlr}

\begin{defn}
    Распределение суммы $k$ квадратов независимых случайных величин со стандартным нормальным распределением называется \textit{распределением хи-квадрат} с $k$ степенями свободы (обозначение: $\chi^{2}_{k}$).
\end{defn}
Плотность распределения $\chi^{2}_{k}$ имеет вид
\begin{equation*}
    f(y) = \begin{cases}
        \cfrac{1}{2^{x / 2} \Gamma(k / 2)} \, x^{\frac{k}{2}-1} e^{-x / 2}, & \text { если } x>0 \\
        0, & \text { если } x \leqslant 0
    \end{cases}
\end{equation*}
\begin{rmrk}
    $\chi^{2}_{2} = \GammaDist_{1/2,1} = \ExpDist_{1/2}$.
\end{rmrk}

\begin{namedthm}[Свойства распределения $\chi^{2}$]\leavevmode
\begin{enumerate}
    \item Если случайные величины $\xi_1 \sim \chi^{2}_{k}$ и $\xi_2 \sim \chi^{2}_{m}$ независимы, то их сумма $\xi_1+\xi_1 \sim \chi^{2}_{k+m}$;
    \item $\Exp \chi^{2}=k, \quad \Var \chi^{2}=2 k$.
    \item Пусть дана последовательность случайных величин $\chi_{n}^{2}$. Тогда при $n \to \infty$:
    \begin{equation*}
        \frac{\chi_{n}^{2}}{n} \xrightarrow[n \to \infty]{p} 1, 
        \quad \frac{\chi_{n}^{2}-n}{\sqrt{2 n}} \xrightarrow[n \to \infty]{\text{d}} \Normal_{0,1}
    \end{equation*}
    \item Пусть случайные величины $\xi_1, \ldots, \xi_n$ независимы и $\xi_i \sim \Normal_{a,\sigma^{2}}$. Тогда
    \begin{equation*}
        \sum\limits_{i=1}^{k}\left(\frac{\xi_{i}-a}{\sigma}\right)^{2} \sim \chi^{2}_{k}.
    \end{equation*}
\end{enumerate}
\end{namedthm}

\begin{defn}
    Пусть $\xi_{0}, \xi_{1}, \ldots, \xi_{k}$ независимы и $\xi_i \sim \Normal_{0,1}$. Распределение случайной величины
    \begin{equation*}
        t_{k}
        = \frac{\xi_{0}}{\sqrt{\frac{\xi_{1}^{2} + \ldots + \xi_{k}^{2}}{k}}} 
        = \frac{\xi_0}{\sqrt{x_{k}^{2} / k}}
    \end{equation*}
    называется \textit{распределением Стьюдента ($t$-распределением} с $k$ степенями свободы ($St(k)$ или $\Student_{k}$).
\end{defn}
Плотность распределения $\Student_{k}$ имеет вид
\begin{equation*}
    f_{k}(y)=\frac{\Gamma\bigl((k+1) / 2\bigr)}{\sqrt{\pi k} \, \Gamma(k / 2)}\left(1+\frac{y^{2}}{k}\right)^{-(k+1) / 2}
\end{equation*}

\begin{namedthm}[Свойства распределения Стьюдента]\leavevmode
\begin{enumerate}
    \item Распределение Стьюдента симметрично, т.е. если $t_k \sim \Student_{k}$, то $-t_k \sim \Student_{k}$.
    \item $\Student_{k} \xrightarrow[k \to \infty]{} \Normal_{0,1}$.
    \item У распределения Стьюдента $\Student_{k}$ существуют только моменты порядка $m < k$, при этом все существующие моменты нечётного порядка равны нулю.
\end{enumerate}
\end{namedthm}

\begin{namedthm}[Теорема Фишера]
Пусть случайные величины $X_1, \ldots, X_n$ независимы и ${X_i \sim \Normal_{a,\sigma^{2}}}$. Тогда:
\begin{enumerate}
    \item $\sqrt{n} \frac{\SampleMean-a}{\sigma} \sim \Normal_{0,1}$
    \item $\frac{(n-1) S_{0}^{2}}{\sigma^{2}}=\sum\limits_{i=1}^{n} \frac{\left(X_{i}-\SampleMean\right)^{2}}{\sigma^{2}} \sim \chi^{2}_{n-1}$
    \item Случайные величины $\SampleMean$ и $S_{0}^{2}$ независимы.
\end{enumerate}
\end{namedthm}
\begin{crlr}
    Пусть случайные величины $X_1, \ldots, X_n$ независимы и ${X_i \sim \Normal_{a,\sigma^{2}}}$. Тогда:
    \begin{enumerate}
        \item $\sqrt{n} \frac{\SampleMean-a}{\sigma} \sim \Normal_{0,1}$ (для $a$ при известном $\sigma^{2}$)
        \item $\sum\limits_{i=1}^{n}\left(\frac{X_{i}-a}{\sigma}\right)^{2} \sim \chi^{2}_{n}$ (для $\sigma^{2}$ при известном $a$)
        \item $\frac{(n-1) S_{0}^{2}}{\sigma^{2}} \sim \chi^{2}_{n-1}$ (для $\sigma^{2}$ при неизвестном $a$)
        \item $\sqrt{n} \frac{\SampleMean-a}{S_{0}} \sim \Student_{n-1}$ (для $a$ при неизвестном $\sigma^{2}$)
    \end{enumerate}
\end{crlr}

\subsubsection{Статистические выводы о параметрах нормального распределения}

Пусть $X_{1}, \ldots, X_{n}$~--- выборка объёма $n$ из распределения $\Normal_{a, \sigma^{2}}$. Построим точные доверительные интервалы (ДИ) с уровнем доверия $\alpha$ для параметров нормального распределения, используя следствие из теоремы Фишера.
\begin{enumerate}
    \item ДИ для $a$ при известном $\sigma^{2}$:
    \begin{equation*}
        \MyPr\left(\SampleMean-\frac{\tau \sigma}{\sqrt{n}}<a<\SampleMean+\frac{\tau \sigma}{\sqrt{n}}\right)=\alpha, ~ \text {где} ~ \varphi_{0,1}(\tau)=\frac{1 + \alpha}{2}
    \end{equation*}
    \item ДИ для $\sigma^{2}$ при известном $a$:
    \begin{equation*}
        \frac{n S_{1}^{2}}{\sigma^{2}} \sim \chi^{2}_{n},~ \text {где}~ S_{1}^{2}=\frac{1}{n} \sum\limits_{i=1}^{n}\left(X_{i}-a\right)^{2}
    \end{equation*}
    Пусть $g_1$ и $g_2$~--- квантили распределения $\chi^{2}_{n}$ уровней $\frac{1-\alpha}{2}$ и $\frac{1+\alpha}{2}$ соответственно. Тогда
    \begin{equation*}
        \frac{(n-1) S_{0}^{2}}{\sigma^{2}} \sim \chi^{2}_{n-1},~ \text {где}~ S_{0}^{2}=\frac{1}{n-1} \sum\limits_{i=1}^{n}\left(X_{i}-\SampleMean\right)^{2}
    \end{equation*}
    \item ДИ для $\sigma^{2}$ при неизвестном $a$:
    \begin{equation*}
        \frac{(n-1) S_{0}^{2}}{\sigma^{2}} \sim \chi^{2}_{n-1},~ \text {где}~ S_{0}^{2}=\frac{1}{n-1} \sum\limits_{i=1}^{n}\left(X_{i}-\SampleMean\right)^{2}
    \end{equation*}
    Пусть $g_1$ и $g_2$~--- квантили распределения $\chi^{2}_{n-1}$ уровней $\frac{1-\alpha}{2}$ и $\frac{1+\alpha}{2}$ соответственно. Тогда
    \begin{equation*}
        \alpha=\MyPr\left(g_{1}<\frac{(n-1) S_{0}^{2}}{\sigma^{2}}<g_{2}\right)=\MyPr\left(\frac{(n-1) S_{0}^{2}}{g_{2}}<\sigma^{2}<\frac{(n-1) S_{0}^{2}}{g_{1}}\right)
    \end{equation*}
    \item ДИ для $a$ при неизвестном $\sigma^{2}$:
    \begin{equation*}
        \sqrt{n} \frac{\SampleMean-a}{S_{0}} \sim \Student_{n-1}
    \end{equation*}
    Пусть $c$~--- квантиль распределения $\Student_{n-1}$ уровня $\frac{1-\alpha}{2}$. Распределение Стьюдента симметрично, поэтому
    \begin{equation*}
        \alpha=\MyPr\left(-c<\sqrt{n} \frac{\SampleMean-a}{S_{0}}<c\right)=\MyPr\left(\SampleMean-\frac{c S_{0}}{\sqrt{n}}<a<\SampleMean+\frac{c S_{0}}{\sqrt{n}}\right)
    \end{equation*}
\end{enumerate}
 % Статистические выводы о параметрах нормального распределения. Распределения хи-квадрат и Стьюдента. Теорема Фишера
