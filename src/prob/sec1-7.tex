\section{Испытания Бернулли. Биномиальное распределение. Теорема Пуассона. Распределение Пуассона}
\begin{defn}
    \textit{Испытание Бернулли}~--- случайный эксперимент, у которого есть ровно два возможных 
    исхода: <<успех>> и <<неудача>>. 
    Как правило, вероятность успеха обозначается буквой $p$, вероятность неудачи $q = 1 - p$.
\end{defn}

\begin{defn}
    \textit{Схема Бернулли}~--- последовательность из $n$ \textit{независимых однородных} испытаний Бернулли c вероятностью успеха $p$ и неудачи $q = 1 - p$.
\end{defn}

Со схемой Бернулли можно связать последовательность случайных величин $\xi_1,\, \xi_2,\, \ldots,\, \xi_n,$ где 
$\xi_k = 
    \begin{cases} 
        1, &\text{с вероятностью } p; \\ 
        0, & \text{с вероятностью } q,
    \end{cases} 
    \quad k = \overline{1, n}$.

Таким образом, принятие случайной величиной $\xi_k$ значения $1$ интерпретируется как успех в $k$-м испытании.
Так как испытания в схеме Бернулли независимы и однородны, данные случайные величины должны быть \textit{независимы и одинаково распределены}.

\begin{namedthm}[Формула Бернулли]
    Пусть $\xi$~--- случайная величина, равная числу успехов в $n$ испытаниях. Тогда $\forall k = \overline{1,n}$ вероятность получить в $n$ испытаниях ровно $k$ успехов равна
    \begin{equation*}
        \MyPr\left(\xi=k\right)=C_{n}^{k} p^{k} q^{n-k}
    \end{equation*}
    \end{namedthm}
    
    \begin{proof}
    Рассмотрим один элементарный исход события $A = \{\xi = k \}$:
    \begin{equation*}
        (\underbrace{y, y, \ldots, y}_{k}, \underbrace{\textit{ н}, \textit{ н}, \ldots,\textit{ н}}_{n-k})
    \end{equation*}
    когда первые $k$ испытаний завершились успехом (у), остальные неудачей (н). Поскольку испытания независимы, вероятность такого элементарного исхода равна $p^k(1 - p)^{n-k}.$ Другие элементарные исходы из события $A$ отличаются лишь расположением $k$ успехов на $n$ местах. Поэтому событие $A$ состоит из $C_n^k$ элементарых исходов, вероятность каждого из которых равна $p^kq^{n-k}$.
    \end{proof}
    
    \begin{defn}
        Пусть $\xi_1, \ldots, \xi_n$~--- последовательность независимых случайных величин, имеющих одинаковое распределение Бернулли с параметром $p$ ($\Bernoulli_{p}$), то есть принимает значение $1$ (<<успех>>) с вероятностью $p$ и $0$ (<<неудача>>) с вероятностью $1 - p = q$. Тогда говорят, что случайная величина $\xi = \xi_1 + \ldots + \xi_n$ имеет \textit{биномиальное распределение} с параметрами $n$ и $p$ ($\Binom_{n, p}$).
    \end{defn}
    
    \subsubsection{Числовые характеристики $\Bernoulli_{p}$}
    \begin{enumerate}
        \item Математическое ожидание:
        \begin{equation*}
            \Exp \xi =  1 \cdot p + 0 \cdot q = p
        \end{equation*}
        \item Дисперсия:
            $$\Exp \xi^2 = 1^2 \cdot p + 0^2 \cdot q = p; \quad \Var\xi = \Exp \xi^2 - (\Exp \xi)^2 = p - p^2 = p \cdot (1 - p) = pq$$
    \end{enumerate}
    
    \subsubsection{Числовые характеристики $\Binom_{n, p}$}
    \begin{enumerate}
        \item Математическое ожидание:
        \begin{equation*}
            \Exp \xi = \Exp (\xi_1 + \ldots + \xi_n) = \Exp \xi_1 + \ldots + \Exp \xi_n = \underbrace{p + \ldots + p}_{n} = np
        \end{equation*}
        \item Дисперсия:
        \begin{equation*}
            \Var\xi = \Var(\xi_1 + \ldots + \xi_n) = \Var\xi_1 + \ldots + \Var\xi_n = \underbrace{p \cdot q + \ldots + p \cdot q}_{n} = npq
        \end{equation*}
    \end{enumerate}
    
    \begin{namedthm}[Теорема Пуассона]
        Пусть проводится $n$ обобщённых испытаний Бернулли (т.е. вероятность успеха испытания зависит от $n$) с вероятностью успеха $p_n$, $\xi$~--- количество успехов в этих испытаниях и $n p_{n} \underset{n \to +\infty}{\longrightarrow} \lambda$. Тогда
        \begin{equation*}
            \forall k \in \Integer,~ 0 \leqslant k \leqslant n: \quad \MyPr\left(\xi=k\right) \underset{n \to +\infty}{\longrightarrow} \frac{\lambda^{k}}{k !} e^{-\lambda}
        \end{equation*}
    \end{namedthm}
    
    \begin{proof}
        По условию теоремы, $n p_n \xrightarrow[n \to +\infty]{} \lambda$. Тогда $p_n \xrightarrow[n \to +\infty]{} 0$. Рассмотрим формулу Бернулли:
        \begin{multline*}
            \MyPr(\xi = k) = C_n^k p^k (1 - p)^{n-k} = \frac{n!}{k!(n - k)!} \cdot \frac{\lambda^k}{n^k} \cdot \left(1 - \frac{\lambda}{n}\right)^{-k} \cdot \left(1 - \frac{\lambda}{n}\right)^n = \\
            = \frac{\lambda^k}{k!} \cdot \frac{(n - k + 1) \cdot \ldots \cdot (n - 1) \cdot n}{n^k} \cdot \left(1 - \frac{\lambda}{n}\right)^n \cdot \left(1 - \frac{\lambda}{n}\right)^{-k}
        \end{multline*}
        Перейдём к пределу при $n \to +\infty$:
        \begin{equation*}
            \frac{(n - k + 1) \cdot \ldots \cdot (n - 1) \cdot n}{n^k} \to 1,~ \left(1 - \frac{\lambda}{n}\right)^n \to e^{-\lambda},~ \left(1 - \frac{\lambda}{n}\right)^{-k} \to 1
        \end{equation*}
        
        Таким образом, получим
        \begin{equation*}
            \MyPr(\xi = k) \xrightarrow[n \to +\infty]{} \frac{\lambda^k}{k!} e^{-\lambda}
        \end{equation*}
    \end{proof}
    
    \begin{defn}
        Набор вероятностей $\{\frac{\lambda^k}{k!} e^{-\lambda} \}$, где $k$ принимает значения $0, 1, 2, \ldots$, называется \textit{распределением Пуассона} с параметром $\lambda > 0$ ($\Pois_{\lambda}$).
    \end{defn}
    \begin{rmrk}
        Распределение Пуассона представляет собой число событий, произошедших за фиксированное время, при условии, что данные события происходят с некоторой фиксированной средней интенсивностью (за которую отвечает параметр $\lambda$) и независимо друг от друга.
    \end{rmrk}
    
    \subsubsection{Числовые характеристики $\Pois_{\lambda}$}
    \begin{enumerate}
        \item Математическое ожидание:
        \begin{align*}
            \sum\limits_{k=0}^{\infty} e^{-\lambda} \frac{\lambda^k}{k!} = e^{-\lambda} \sum\limits_{k=0}^{\infty} \frac{\lambda^k}{k!} = e^{-\lambda} e^\lambda = 1 \\
            \Exp \xi = \sum\limits_{k=1}^{\infty} k e^{-\lambda} \frac{\lambda^k}{k!} = e^{-\lambda} \lambda \underbrace{\sum\limits_{k=1}^{\infty} \frac{\lambda^{k-1}}{(k - 1)!}}_{= e^\lambda} = \lambda
        \end{align*}
        \item Дисперсия:
        \begin{align*}
            \Exp \xi(\xi - 1) = \sum\limits_{k=0}^{\infty} k (k - 1) \frac{\lambda^k}{k!} e^{-\lambda}  = \lambda^2 e^{-\lambda} \sum\limits_{k=2}^{\infty} \frac{\lambda^{k-2}}{(k-2)!} = \lambda^2 e^{-\lambda} e^\lambda = \lambda^2 \\
            \Exp \xi^2 = \Exp \xi(\xi - 1) + \Exp \xi = \lambda^2 + \lambda \quad \implies \quad \Var\xi = \Exp \xi^2 - (\Exp \xi)^2 = \lambda
        \end{align*}
    \end{enumerate}