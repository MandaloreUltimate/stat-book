\section{Испытания Бернулли. Теорема Муавра"--~Лапласа. Нормальное распределение}

\begin{namedthm} [Локальная предельная теорема Муавра"--~Лапласа]
    Пусть $S_n$~--- число успехов в $n$ испытаниях Бернулли с вероятностью успеха $0 < p < 1$. 
    Пусть $n \to \infty$, тогда $n p(1-p) {\longrightarrow} \infty$, и 
    \begin{equation*}
        \forall m \in \Integer: 0 \leqslant m \leqslant n \quad \MyPr\left(S_{n}=m\right)=\frac{1}{\sigma \sqrt{2 \pi} } e^{-\frac{x^{2}}{2}}\left(1+\underline{O}\left(\frac{1}{\sigma}\right)\right),
    \end{equation*}
    где $x = \frac{m - np}{\sigma},$ а $\sigma=\sqrt{\Var S_{n}}=\sqrt{n p(1-p)}$.
\end{namedthm}  

\begin{namedthm}[Интегральная теорема Муавра"--~Лапласа]
Если выполнено условие локальной теоремы и $C$~--- произвольная положительная константа, то равномерно по $a$ и $b$ из отрезка $[-C,C]$ (пусть $b \geqslant a$)
\begin{equation*}
    \MyPr\left(a \leqslant \frac{S_{n}-n p}{\sqrt{n p(1-p)}} \leqslant b\right) \xrightarrow[n \to +\infty]{} \frac{1}{\sqrt{2 \pi}} \int\limits_{a}^{b} e^{-\frac{x^{2}}{2}} dx.
\end{equation*}
\end{namedthm} 

\begin{defn}
    Случайная величина $\xi$ имеет \textit{нормальное (гауссовское) распределение} с параметрами $a$ и $\sigma^2$ ($\Normal_{a, \sigma^2}$), где $a \in \Real, \sigma > 0$, если $\xi$ имеет следующую плотность распределения: 
\begin{equation*}
    f(x)=\frac{1}{\sigma \sqrt{2 \pi}} e^{-\frac{(x-a)^{2}}{2 \sigma^{2}}}, \quad x \in \Real.
\end{equation*}
\end{defn}

\subsubsection{Числовые характеристики $\Normal_{a, \sigma^2}$}

Найдем матожидание и дисперсию для \textit{стандартного} нормального распределения ($\alpha = 0, \, \sigma^2 = 1$):
\begin{multline*}
    \Exp \xi = 
    \int\limits_{-\infty}^{\infty} x f_{\xi}(x) dx =
    \frac{1}{\sqrt{2\pi}} \int\limits_{-\infty}^{\infty} x e^{\frac{-x^2}{2}} dx = 
    -\frac{1}{\sqrt{2\pi}} \int\limits_{-\infty}^{\infty} d\left( e^{\frac{-x^2}{2}}\right) = 
    \left. -e^{\frac{-x^2}{2}}\right|_{-\infty}^{\infty} = 0 \\
    \Exp \xi^{2}=\frac{1}{\sqrt{2 \pi}} \int\limits_{-\infty}^{\infty} x^{2} e^{-x^{2} / 2} d x=\frac{2}{\sqrt{2 \pi}} \int\limits_{0}^{\infty} x^{2} e^{-x^{2} / 2} d x=-\frac{2}{\sqrt{2 \pi}} \int\limits_{0}^{\infty} x d e^{-x^{2} / 2}= \\
    =-\left.\frac{2 x}{\sqrt{2 \pi}} e^{-x^{2} / 2}\right|_{0} ^{\infty}+2 \int\limits_{0}^{\infty} \frac{1}{\sqrt{2 \pi}} e^{-x^{2} / 2} d x=0+\int\limits_{-\infty}^{\infty} \frac{1}{\sqrt{2 \pi}} e^{-x^{2} / 2} d x=1.
\end{multline*}

Поэтому $\Var\xi = \Exp \xi^2 - (\Exp \xi)^2 = 1 - 0 = 1.$

Теперь рассмотрим случайную величину $\eta$ с нормальным распределением в общем виде (с параметрами $\alpha$ и $\sigma^2$). Тогда $\xi = \frac{\eta - \alpha}{\sigma}$ - случайная величина со \textit{стандартным} нормальным распределением. Далее, т.к. $\Exp \xi = 0$, $\Var\xi = 1$, то 
\begin{gather*}
    \Exp \eta = \Exp (\sigma \xi + \alpha) = \sigma \Exp \xi + \alpha = \alpha, \\
    \Var\eta = \Var(\sigma \xi + \alpha) = \sigma^2 \Var\xi = \sigma^2.
\end{gather*}
%sec1-9
\begin{center}
    \begin{tikzpicture}[
        declare function={
            normal_pdf(\x,\sstd,\mu) = 1 / sqrt(2 * pi * \sstd) * exp(-(\x - \mu)^2 / (2 * \sstd));
            binom_pmf(\x,\n,\p) = \n! / (\x! * (\n - \x)!) * \p^(\x) * (1 - \p)^(\n - \x);
            pois_pmf(\x,\l) = \l^(\x) * exp(-\l) / (\x!);
        }]
        \begin{axis}[
            height=11cm, width=18cm,
            xmin=0, xmax=11,
            ymin=-0.05, ymax=0.45,
            xlabel={$x$},
            ylabel={$f(x)$},
            xtick={1, ..., 10},
            ytick={0.1, 0.2, ..., 0.4},
            axis line style=thick,
            axis lines=middle,
            enlargelimits=false
            ]
            \addplot[very thick, color1, samples=100,domain=0:10] {normal_pdf(x,2.5,5)};
            \addplot[color=red, domain=0:10, samples=11, mark=*] {binom_pmf(x,10,0.5)};
            \addplot[color=color2, domain=0:10, samples=11, mark=*] {pois_pmf(x,1)};
            \addplot[color=blue, domain=0:10, samples=11, mark=*] {pois_pmf(x,3)};
            \addplot[color=black, domain=0:10, samples=11, mark=*] {pois_pmf(x,5)};
            \legend{$\Normal_{5;2.5}$, $\Binom_{10;0.5}$, $\Pois_{1}$, $\Pois_{3}$, $\Pois_{5}$}
        \end{axis}
    \end{tikzpicture}
\end{center}
\vspace{1cm}
\begin{rmrk}
Формулу из теоремы Муавра"--~Лапласа можно переписать в следующем виде:
\begin{equation*}
    \MyPr\left(a \leqslant \frac{S_{n}-n p}{\sqrt{n p(1-p)}} \leqslant b\right) \xrightarrow[n \to +\infty]{} \Phi(b) - \Phi(a),
\end{equation*}
где $\Phi(x)=\frac{1}{\sqrt{2 \pi}} \int\limits_{-\infty}^{x} e^{-t^{2} / 2} d t$~--- функция распределения стандартного нормального распределения.
\end{rmrk}