\section{Дискретные, сингулярные и абсолютно непрерывные функции распределения и случайные величины. Плотность распределения. Теорема Лебега о разложении функции распределения}

\begin{defn}
    Распределение $\xi$ называется \textit{дискретным}, если существует не более чем счётное множество $B$, т.ч. $P_\xi(B) = 1$. 
    \textit{Дискретная функция распределения} имеет вид:
    \begin{equation*}
        F_\xi(x) = \MyPr(\xi < x) = \sum\limits_{x_i < x}{}p_{i} = \sum\limits_{x_i < x}{}\MyPr(\xi = x_{i}).
    \end{equation*}
\end{defn}

\begin{rmrk}
    Для любой дискретной функции распределения $F_\xi(x)$ число скачков~--- не более чем счётное.
    
    Действительно, можно перенумеровать все скачки следующим образом:
    \begin{equation*}
        \Delta_{n}=\left\{t \colon F_{x}(t+0)-F_{x}(t)>\frac{1}{n}\right\},~ |\Delta_{n} | \leqslant n.
    \end{equation*}
    
    Т.е. на каждом шаге мы считаем все скачки величины более $1 / n$, а таких скачков не больше, чем $n$, так как функция распределения ограничена снизу нулём, сверху единицей, и, кроме того, монотонна.
    Множество точек разрыва представимо в виде $\bigcup\limits_{n = 1}^{\infty} \Delta_{n}$, т.е. не более чем счётно.
\end{rmrk}

\begin{defn}
    Распределение $\xi$ называется \textit{абсолютно непрерывным}, если существует $f(x) \overset{\text{п.н.}}{\geqslant} 0$ такая, что для любого борелевского множества $B$ справедливо
    \begin{equation*}
        P_\xi(B) = \int\limits_B f(x) \lambda(dx),
    \end{equation*}
    где $f(x)$~--- \textit{плотность распределения}, $\lambda$~--- мера Лебега. 
    \textit{Абсолютно непрерывная функция распределения} имеет вид:
    \begin{equation*}
        F_\xi(x) = \MyPr(\xi < x) = \int\limits_{-\infty}^x f(t)dt.
    \end{equation*}
\end{defn}

\begin{rmrk}
\begin{enumerate}\leavevmode
    \item 
        $f(x) \overset{\text{п.н.}}{\geqslant} 0$ (почти наверное неотрицательна), если множество точек, где это неравенство не выполняется, имеет меру нуль по Лебегу, т.е. $\MyPr(f(x) < 0) = 0$.
    \item 
        В определении абсолютно непрерывного распределения стоит не интеграл Римана, а его обобщение, \textit{интеграл Лебега}.
    \item 
        В некоторых вариантах определения от плотности требуется неотрицательность не почти наверное, а всюду на $\Real$. 
        Это вопрос соглашения, так как интеграл Лебега по множеству меры нуль в любом случае равен нулю.
    \item 
        В случае абсолютно непрерывного распределения вероятность попасть в конкретную точку равна нулю. Действительно,
        \begin{equation*}
            \MyPr(\xi = x) = \int\limits_{x}^{x} f(t)dt = 0.
        \end{equation*}
\end{enumerate}
\end{rmrk}

\begin{namedthm}[Свойства плотности]\leavevmode
\begin{enumerate}
    \item $f_{\xi}(x) = \frac{d}{dx}F(x)$ почти всюду (кроме, может быть, множества меры нуль по Лебегу~--- например, функция равномерного распределения $\Uniform_{[0, 1]}$ в точках $0$ и $1$ не дифференцируема);
    \item $\int\limits_{-\infty}^{+\infty} f_\xi(t) dt = 1$ (нормировка).
\end{enumerate}
\end{namedthm}
\begin{proof}
    Первое свойство очевидно из свойств интегралов с переменным верхним пределом, рассмотрим второе. 
    Если в определении абсолютно непрерывного распределения в качестве борелевского множества взять всю числовую прямую, получим: 
    \begin{equation*}
        \MyPr(\xi \in \Real)=1=\int\limits_{\Real} f_{\xi}(x) dx.
    \end{equation*}
\end{proof}

\begin{defn}
    \textit{Точка роста} функции распределения $F_\xi(x)$~--- точка $x_0$, для которой справедливо:
\begin{equation*}
    \forall \varepsilon > 0 \quad F_\xi(x_0 + \varepsilon) - F_\xi(x_0 - \varepsilon) > 0
\end{equation*}
\end{defn}

\begin{rmrk}
    Возможен случай, когда точка роста является точкой разрыва:
    \begin{multline*}
    \lim _{\varepsilon \to 0} \left( F_{\xi}\left(x_{0}+\varepsilon\right) - F_{\xi}\left(x_{0}-\varepsilon\right) \right)=F_{\xi}\left(x_{0}+0\right)-F_{\xi}\left(x_{0}\right)>0 \iff \\
    \iff \lim\limits_{\varepsilon \to 0}\MyPr\left(x_{0}-\varepsilon \leqslant \xi<x_{0}+\varepsilon\right)=\MyPr\left(\xi=x_{0}\right)>0
    \end{multline*}
\end{rmrk}

\begin{defn}
    Функция распределения $F_\xi(x)$ называется \textit{сингулярной}, если она непрерывна и множество точек её роста имеет нулевую меру Лебега.
\end{defn}
\begin{exmp}
    Сингулярной функцией является \textit{лестница Кантора} \\ $c \colon [0, 1] \mapsto [0, 1]$, которая строится следующим образом:
    
    $c(0) = 0, c(1) = 1$. 
    Далее интервал $(0, 1)$ разбивается на три равные части $(0, 1/3)$, $(1/3, 2/3)$, $(2/3, 1)$. 
    На среднем интервале полагаем $c(x) = 1/2$, оставшиеся два интервала снова разбиваются на три равные части каждый, и на соответствующих средних интервалах полагаем $c(x) = 1/4$ и $c(x) = 3/4$. 
    Каждый из оставшихся интервалов снова делится на три части, и на внутренних интервалах $c(x)$ определяется как постоянная, равная среднему арифметическому между соседними, уже определенными значениями $c(x)$. 
    На остальных точках единичного отрезка определяется по непрерывности. 
\end{exmp}

\begin{rmrk}
    Так как любая функция распределения дифференцируема почти всюду, возможность дифференцировать функцию распределения никакого отношения к существованию плотности не имеет. 
    Даже если мы дополнительно потребуем непрерывности функции распределения, этого не будет достаточно для абсолютной непрерывности распределения. 
    К примеру, функция распределения сингулярного распределения непрерывна и дифференцируема почти всюду, однако плотности у этого распределения нет, так как производная функции распределения почти всюду равна нулю.
\end{rmrk}

\begin{namedthm}[Теорема Лебега о разложении функции распределения]
    Пусть $\xi$~--- случайная величина с функцией распределения $F_\xi(x).$ 
    Тогда существуют и определены единственным образом три функции распределения $F_{ac}(x), F_s(x), F_d(x)$, 
    абсолютно непрерывная, сингулярная и дискретная соответственно, а также три числа $p_1, p_2, p_3 \geqslant 0,\, p_1 + p_2 + p_3 = 1$ такие, что 
    \begin{equation*}
        F_{\xi}(x)=p_{1} F_{ac}(x)+p_{2} F_{s}(x)+p_{3} F_{d}(x).
    \end{equation*}
\end{namedthm}
