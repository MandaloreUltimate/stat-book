\documentclass[oneside,final,14pt]{extreport}

%Общий вид и оформление
\usepackage{setspace}
\usepackage{indentfirst}
\usepackage{geometry}
\geometry{
  a4paper,
  left=20mm,top=15mm,right=20mm,bottom=15mm,
  headheight=0pt,headsep=0mm,foot=0pt,footskip=13mm,
  includeheadfoot
}
\linespread{1.05}
\raggedbottom
\sloppy

%Русский язык
\usepackage[nottoc,notlot,notlof]{tocbibind}
\usepackage{cmap}
\usepackage[T2A]{fontenc}
\usepackage[utf8]{inputenc}
\usepackage[english, russian]{babel}

%Шрифты
\usepackage{fix-cm}
\usepackage{microtype}
\usepackage{anyfontsize}
\newcommand\veryhuge{\fontsize{55}{66}\selectfont}
\newcommand\verylarge{\fontsize{25}{30}\selectfont}
\newcommand\quitelarge{\fontsize{22.5}{27}\selectfont}
\newcommand{\garamond}{\fontencoding{T2A}\fontfamily{CormorantGaramond-LF}\selectfont}
\newcommand{\gillius}{\fontencoding{T1}\fontfamily{GilliusADFNoTwo-LF}\selectfont}

%Математические формулы
\usepackage{amsmath}
\usepackage{amsthm}
\usepackage{amssymb}
\usepackage{centernot}
\allowdisplaybreaks

\newcommand{\notimplies}{\centernot\implies}
\newcommand{\notimpliedby}{\centernot\impliedby}

%Таблицы
\usepackage{adjustbox}
\usepackage{hhline}
\usepackage{multirow}
\usepackage{caption}

\newcommand\mylinespace[1][10pt]{\rule[\normalbaselineskip]{0pt}{#1}}
\newcommand{\doublerow}[2]{\begin{tabular}{@{}c@{}}#1 \\ #2\end{tabular}}

\DeclareCaptionType{mytype}[Таблица][Список] %Название таблицы без окружения table
\newenvironment{mytable}{\captionsetup{type=mytype}}{}

%Графика
\usepackage[dvipsnames,svgnames]{xcolor}
\usepackage{graphicx}
\usepackage{tikz}
\usepackage{tikz-cd}
\usepackage{pgfplots}
\pgfplotsset{compat=1.16}
\usepgfplotslibrary{fillbetween}

%Заголовки глав и секций
\usepackage{titlesec}   
\titleformat{\section}
  {\large\bfseries}{\thesection}{1em}{}   
\titleformat{\chapter}
    {\Huge\bfseries}{}{0pt}{}
\titlespacing*{\chapter}
    {0cm}{-\topskip}{1em}[0pt]
    
%Библиография
\usepackage{csquotes}
\usepackage[
    backend=biber, 
    sorting=nyt,
    bibstyle=gost-authoryear,
    citestyle=gost-authoryear
]{biblatex}
\addbibresource{src/refs.bib}

%Гиперссылки
\usepackage{hyperref}
\hypersetup{
    colorlinks,
    citecolor=black,
    filecolor=black,
    linkcolor=blue,
    urlcolor=blue
}

%Прочие пакеты
\usepackage{relsize}
\usepackage{enumitem}
\usepackage[perpage]{footmisc}

%Обозначения математических операторов
\DeclareMathOperator{\Exp}{\mathbb{E}}
\DeclareMathOperator{\Var}{\mathbb{D}}
\DeclareMathOperator{\MyPr}{\mathbb{P}}

\DeclareMathOperator{\ExpTh}{\Exp_{\theta}}
\DeclareMathOperator{\VarTh}{\mathbb{D}_{\theta}}
\DeclareMathOperator{\MyPrTh}{\MyPr_{\theta}}

\DeclareMathOperator{\Ind}{\mathrm{I}}

\DeclareMathOperator{\Dist}{\mathcal{P}}
\DeclareMathOperator{\Uniform}{\mathrm{U}}
\DeclareMathOperator{\Bernoulli}{\mathrm{B}}
\DeclareMathOperator{\Binom}{\mathrm{Bi}}
\DeclareMathOperator{\NegBinom}{\mathrm{NB}}
\DeclareMathOperator{\Geom}{\mathrm{Geom}}
\DeclareMathOperator{\ExpDist}{\mathrm{Exp}}
\DeclareMathOperator{\Pois}{\mathrm{Pois}}
\DeclareMathOperator{\Normal}{\mathrm{N}}
\DeclareMathOperator{\Cauchy}{\mathrm{C}}
\DeclareMathOperator{\GammaDist}{\Gamma}
\DeclareMathOperator{\Student}{\mathrm{T}}

\newcommand{\Real}{\mathbb{R}}
\newcommand{\Complex}{\mathbb{C}}
\newcommand{\Integer}{\mathbb{Z}}
\newcommand{\Natural}{\mathbb{N}}
\newcommand{\Rational}{\mathbb{Q}}
\newcommand{\Irrational}{\mathbb{I}}

\newcommand{\Alg}{\mathcal{A}}
\newcommand{\SigAlg}{\mathcal{F}}
\newcommand{\Borel}{\mathfrak{B}}

\newcommand{\Sample}{\bigl(X_1, \ldots, X_n \bigr)}
\newcommand{\SampleX}{\mathbf{X}}
\newcommand{\SampleMean}{\overline{X}}
\newcommand{\SampleVar}{S^{2}}
\newcommand{\AllTh}{\forall \theta \in \Theta} 

%Оформление определений, теорем, доказательств и т.п.
\renewcommand{\qedsymbol}{$\blacksquare$}
\renewenvironment{proof}{{\bfseries Доказательство.}}{\qed}

\theoremstyle{definition}
\newtheorem*{defn}{Определение}
\newtheorem*{exmp}{Пример}
\newtheorem*{symb}{Обозначение}
\newtheorem*{rmrk}{Замечание}
\newtheorem*{task}{Задача}

\theoremstyle{plain}
\newtheorem*{thm*}{Утверждение}
\newtheorem*{lem}{Лемма}
\newtheorem*{crlr}{Следствие}

\newtheoremstyle{named}{}{}{\itshape}{}{\bfseries}{.}{.5em}{\thmnote{#3}}
\theoremstyle{named}
\newtheorem*{namedthm}{Теорема}

%Компактный список
\newcommand\sbullet[1][.5]{\mathbin{\vcenter{\hbox{\scalebox{#1}{$\bullet$}}}}}
\newenvironment{compactlist}{
    \begin{list}{{$\sbullet[.75]$}}
    {\setlength\partopsep{0pt}
     \setlength\parskip{0pt}
     \setlength\parsep{0pt}
     \setlength\topsep{0pt}
     \setlength\itemsep{0pt}
    }}
    {\end{list}}

\newcommand{\CoverName}{Cover} %Нумерация обложки